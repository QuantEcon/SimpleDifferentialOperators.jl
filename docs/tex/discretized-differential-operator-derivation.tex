% !TEX program = pdflatex

\documentclass[11pt]{article}
\usepackage{amsmath,amsfonts,amsthm,amssymb,geometry,dsfont}
\usepackage[usenames,dvipsnames,svgnamesable]{xcolor}
\usepackage[capitalise,noabbrev]{cleveref} %
\usepackage{natbib,url}
\crefname{equation}{}{} %
\crefname{assumption}{Assumption}{Assumptions}
\crefname{property}{Property}{Properties}
\geometry{left=1in,right=1in,top=0.6in,bottom=1in}

\newcommand{\D}[1][]{\ensuremath{\boldsymbol{\partial}_{#1}}}
\newcommand{\R}{\ensuremath{\mathbb{R}}}
\newcommand{\diff}{\ensuremath{\mathrm{d}}}
\newcommand{\set}[1]{\ensuremath{\left\{{#1}\right\}}}
\newcommand{\indicator}[1]{\ensuremath{\mathds{1}\left\{{#1}\right\}}}
\newcommand{\condexpec}[3][]{\ensuremath{\mathbb{E}_{#1}\left[{#2} \; \middle| \; {#3} \right]}}
\newcommand{\expec}[2][]{\ensuremath{\mathbb{E}_{{#1}}\left[ {#2} \right]}}
\geometry{left=1in,right=1in,top=0.6in,bottom=1in}
\newenvironment{psmallmatrix}
{\left(\begin{smallmatrix}}
	{\end{smallmatrix}\right)}

\theoremstyle{definition}
\newtheorem{example}{Examples}[section]

\bibliographystyle{ecta}
\begin{document}
\title{Derivations and Applications for \url{SimpleDifferentialOperators.jl}}
\author{Chiyoung Ahn and Jesse Perla}
\maketitle

\section{Overview and Notation}
This package is intended to be a stop-gap while more complete and higher-performance solutions are implemented (in particular, the evolution of \url{https://github.com/JuliaDiffEq/DiffEqOperators.jl/}).  In the meantime, the package hopefully provides a solution for discretizing operators and solving ODEs/PDEs.\footnote{Another consideration here is that in the current form, composing operators as sparse matrices, as the package currently does, will take longer than a hand-tweaked model creating composed operators directly.  In many cases, this wouldn't be a problem, but in some algorithms where operators need to be redefined frequently in tight loops, it might be.  In those cases, use the output of this package for test-suites on hand-built discretizations.}

The focus is on discretizing linear (and, with some careful checks, affine) operators.  In the case of models where the process is non-linear, it is most appropriate for algorithms that iteratively solve linear ODEs/PDEs.

\subsection{Linear Differential Equations}



\paragraph{Boundary Conditions}
The package supports some key boundary conditions used for stochastic processes and ODE/PDEs.

As will become clear in the discretization, whether the boundary condition is homogenous or not (i.e. $=0$ or $=b > 0$) is important for the numerical methods.   To detail a few of the one-dimensional versions of the supported boundary conditions

\begin{itemize}
	\item Reflecting Barriers (i.e. a homogeneous Neumann Boundary Conditions)
	\begin{align}
		\D[x]\tilde{v}({x_{\min}} ) &= 0\label{eq:reflecting-BC1}\\
		\D[x]\tilde{v}({x_{\max}}) &= 0\label{eq:reflecting-BC2}
	\end{align}
	\item Mixed boundary conditions (i.e a homogeneous Robin Boundary Conditions):
	\begin{align}
	\underline{\xi} \tilde{v}({x_{\min}}) + \D[x]\tilde{v}({x_{\min}} ) &= 0\label{eq:mixed-BC1}\\
	\overline{\xi} \tilde{v}({x_{\max}}) + \D[x]\tilde{v}({x_{\max}}) &= 0\label{eq:mixed-BC2}
	\end{align}
	Note that when $\underline{\xi} = \overline{\xi} = 0$, this nests the reflecting barriers
	\item Absorbing Barriers (i.e. a homogenous or inhomogeneous Dirichlet Boundary Conditions)
	\begin{align}
		\tilde{v}({x_{\min}} ) &= b_1\label{eq:absorbing-BC1}\\
		\tilde{v}({x_{\max}}) &= b_2\label{eq:absorbing-BC2}
	\end{align}
	In the case of $b_1 = b_2 = 0$, this is homogeneous
\end{itemize}
Of course, many models would have different boundary conditions on different sides of the domain.


\subsection{Discretization Notation}
This section defines the grids and other notation for the discretization.
\begin{itemize}
	\item Define an irregular grid $\set{x_i}_{i=0}^{M+1}$ with \textbf{boundary nodes}, $x_0 = {x_{\min}}$ and $x_{M+1} = {x_{\max}}$. Denote the \textbf{extended grid} as $\overline{x} \equiv \set{x_i}_{i=0}^{M+1}$ and the \textbf{interior grid}, a collection of nodes excluding the boundary nodes, as $x \equiv \set{x_i}_{i=1}^{M} $.

	\item Define $v$ as a vector of a function of interest, $\tilde{v}(x)$, evaluated on the interior grid $x$, i.e., $v = \set{\tilde{v}(x_i)}_{i=1}^M$. Likewise, we define a vector of $v$ on the extended grid as $\overline{v} = \set{\tilde{v}(\overline{x}_i)}_{i=0}^{M+1}$.
	\item Denote the backward and forward distance between the grid points as
	\begin{align}
	\Delta_{i,-} &\equiv x_i - x_{i-1},\, \text{for } i = 1,\ldots, M+1\\
	\Delta_{i,+} &\equiv x_{i+1} - x_i,\, \text{for } i = 0,\ldots, M
	\end{align}
	\item Define the vector of backwards and forwards first differences, padding with $\Delta_{0,-} = \Delta_{M+1,+} = 0$, as
	\begin{align}
	\Delta_{-} &\equiv \begin{bmatrix} 0 \\
	\text{diff}(z)
	\end{bmatrix}\in\R^{M+2}\label{eq:Delta-minus}\\
	\Delta_{+} &\equiv \begin{bmatrix} \text{diff}(z)\\
	0
	\end{bmatrix}\in\R^{M+2}
	\end{align}

\end{itemize}

Let $L_{1-}$, $L_{1+}$ be the discretized backward and forward first order differential operators and $L_2$ be the discretized central difference  second order differential operator, all subject to the Neumann boundary conditions in \cref{eq:mixed-BC1,eq:mixed-BC2}, such that $L_{1-} \bar{v}, L_{1+} \bar{v}$ and $L_2 \bar{v}$ represent the first-order (backward and forward) and second-order derivatives of $\tilde{v}(x)$ respectively at $x$.
\section{Discretizing Operators with a Regular Grid}
In this section, we study discretization schemes under regular grids, i.e., grids such that $x_{i+1} - x_i = \Delta$ for all $i = 0,...,M$ for some fixed $\Delta > 0$.
\subsection{Extension Operators}
Consider constructing first-order derivatives by backward difference $i$th node:
\begin{equation}\label{eq:first-order-bd}
v'_{i,-} = \dfrac{v_{i} - v_{i-1}}{\Delta}
\end{equation}
similarly, using forward differences,
\begin{equation}\label{eq:first-order-fd}
v'_{i,+} = \dfrac{v_{i+1} - v_{i}}{\Delta}
\end{equation}
and second-order derivatives,
\begin{equation}\label{eq:second-order}
v''_{i} = \dfrac{v_{i+1} - 2 v_{i} + v_{i-1} }{\Delta^2}
\end{equation}
for $i \in \{1, ..., M\}$. Stacking over all first-order derivatives and second order derivatives as vectors,
\begin{align}
v'_{-} &= \set{v'_{i,-}}_{i=1}^M \\
v'_{+} &= \set{v'_{i,+}}_{i=1}^M \\
v'' &= \set{v''_{i}}_{i=1}^M
\end{align}
using \eqref{eq:first-order-bd}, \eqref{eq:first-order-fd}, \eqref{eq:second-order}, one can represent the vectors of discretized derivatives as
\begin{align}
v'_{-} &= L_{1-} \overline{v} \\
v'_{+} &= L_{1+} \overline{v} \\
v'' &=  L_{2} \overline{v}
\end{align}
with the following \textbf{extension operators}:
\begin{align}
L_{1-} &\equiv \frac{1}{\Delta}\begin{pmatrix}
-1 &1 &0&\dots&0&0&0&0\\
0&-1&1&\ddots&0&0&0&0\\
\vdots&\vdots&\vdots&\ddots&\vdots&\vdots&\vdots&\vdots\\
0&0&0&\dots&-1&1&0&0\\
0&0&0&\cdots&0&-1&1&0
\end{pmatrix}_{M\times (M+2)}\label{eq:L-1-extended-regular} \\
L_{1+} &\equiv \frac{1}{\Delta}\begin{pmatrix}
0&-1&1&0&\dots&0&0&0\\
0&0&-1&1&\dots&0&0&0\\
\vdots&\vdots&\vdots&\ddots&\ddots&\vdots&\vdots&\vdots\\
0&0&0&0&\dots&-1&1&0\\
0&0&0&0&\cdots&0&-1&1
\end{pmatrix}_{M\times( M+2)}\label{eq:L-1-plus-extended-regular} \\
L_2 &\equiv \frac{1}{\Delta^2}\begin{pmatrix}
1&-2  &1&0&\dots&0&0&0&0\\
0&1&-2&1&\dots&0&0&0&0\\
\vdots&\vdots&\vdots&\vdots&\ddots&\vdots&\vdots&\vdots&\vdots\\
0&0&0&0&\dots&1&-2&1&0\\
0&0&0&0&\cdots&0&1&-2 &1
\end{pmatrix}_{M\times (M+2)}\label{eq:L-2-extended-regular}
\end{align}

which can be defined in more compact forms:
\begin{align}
L_{1-} &\equiv  \Delta^{-1} \begin{bmatrix}
\text{tridiag}(\mathbf{0}_{M-1}, -\mathbf{1}_M, \mathbf{1}_{M} ) & \mathbf{0}_M
\end{bmatrix} \\
L_{1+} &\equiv \Delta^{-1} \begin{bmatrix}
\mathbf{0}_M & \text{tridiag}(\mathbf{0}_{M-1}, -\mathbf{1}_M, \mathbf{1}_{M} )
\end{bmatrix} \\
L_2 &\equiv \frac{1}{\Delta^2} \text{tridiag}^+(\mathbf{1}_{M}, -2\mathbf{1}_M, \mathbf{1}_{M} )
\end{align}

where $\text{tridiag}(x,y,z)$ is a matrix whose lower, main, upper diagonal vectors are $x$, $y$, and $z$, respectively, and $\text{tridiag}^+(x,y,z)$ is a matrix whose main, upper, and second upper diagonal vectors $x$, $y$, and $z$, respectively.

\subsection{Applying Boundary Conditions}
Boundary conditions can be applied manually by using operators on extended grids, $\overline{x}$, to find solutions on extended grids.
Suppose that we want to solve a system $\tilde{L} \tilde{v}({x}) = \tilde{f}(x) $ for $\tilde{v}(x)$ where $L$ is a linear combination of discretized differential operators for some $f$ that represents the values of a function $\tilde{f}(\cdot)$ on discretized interior $x$. To solve the system under boundary conditions on $v$, one can construct and solve the following extended system:

\begin{equation}\label{eq:reflecting-barrier-extended-system}
\begin{bmatrix}
L \\
B
\end{bmatrix}
\overline{v} =
\begin{bmatrix}
f \\
b
\end{bmatrix}
\end{equation}

with $M_E$ by $(M+2)$ matrix $B$ and $M_E$-length vector $b$ that represent the current boundary conditions, where $M_E$ is the number of boundary conditions to be applied. The solution of \eqref{eq:reflecting-barrier-extended-system}, $\tilde{v}(\overline{x})$ can be decomposed into
\begin{align}
\overline{v} = \begin{bmatrix}
\tilde{v}(x_0) \\
v \\
\tilde{v}(x_{M+1})
\end{bmatrix}
\end{align}
which also gives the solution for $v$.

\subsubsection{Mixed Boundary Conditions}
Recall mixed boundary conditions from \eqref{eq:mixed-BC1} and \eqref{eq:mixed-BC2}. Note that reflecting barrier conditions are special cases with $\overline{\xi} = \underline{\xi} = 0$. Using forward difference and backward difference discretization scheme for the lower bound and upper bound respectively, we have
\begin{align}
&\frac{\overline{v}_1 - \overline{v}_0}{\Delta} - \underline{\xi} \overline{v}_0 = 0 \\
&\frac{\overline{v}_{M+1} - \overline{v}_M}{\Delta} - \overline{\xi} \overline{v}_{M+1} = 0
\end{align}

Thus, the corresponding boundary condition matrix $B$ is

\begin{equation}\label{eq:mixed-barrier-matrix-original-regular}
B = \begin{bmatrix}
-\dfrac{1}{\Delta} + \underline{\xi} & \dfrac{1}{\Delta} & 0 & \dots & 0 & 0 & 0 \\
0 & 0 & 0 & \dots & 0 & -\dfrac{1}{\Delta} & \dfrac{1}{\Delta} + \overline{\xi}\\
\end{bmatrix}_{2 \times (M+2)} \quad
b = \begin{bmatrix}
0 \\
0
\end{bmatrix}
\end{equation}
which provides the identical system as
\begin{equation}\label{eq:mixed-barrier-matrix-regular}
B = \begin{bmatrix}
-1 +  \underline{\xi} \Delta & 1 & 0 & \dots & 0 & 0 & 0 \\
0 & 0 & 0 & \dots & 0 & -1 & 1 + \overline{\xi} \Delta\\
\end{bmatrix}_{2 \times (M+2)} \quad
b = \begin{bmatrix}
0 \\
0
\end{bmatrix}
\end{equation}

Subtracting the first row of $\overline{L}_{1-}$ by the first row of $B$ in \eqref{eq:mixed-barrier-matrix} multiplied by $(-1 + \underline{\xi} \Delta_{1,-} )^{-1}\Delta_{1,-}^{-1}$ gives, with the corresponding row operation matrix $R$,

\begin{equation}
R {L}_{1-} = \begin{pmatrix}
1 - (1 + \underline{\xi}\Delta )&0&0&\dots&0&0&0\\
-1&1&0&\ddots&0&0&0\\
\vdots&\vdots&\vdots&\ddots&\vdots&\vdots&\vdots\\
0&0&0&\dots&-1&1&0\\
0&0&0&\cdots&0&-1&1
\end{pmatrix}_{M\times M}
\end{equation}
note that there is no zero element in the first and last column for nodes on boundaries. Hence, solving the corresponding extended system,
\begin{align}
\begin{bmatrix}
{L}\\
B
\end{bmatrix} =
\begin{bmatrix}
f \\ b
\end{bmatrix}
\end{align}
is identical as solving the following system
\begin{align}
R
\begin{bmatrix}
{L}\\
B
\end{bmatrix} &=
R
\begin{bmatrix}
f \\ b
\end{bmatrix} \\
&= \begin{bmatrix}
f \\ b
\end{bmatrix}
\end{align}
as $b$ is a zero vector so that the row operations $R$ do not change anything on the RHS. Furthermore, limited to the interior, solving $v$ in the system above is identical as solving the following system with an operator $L^B$ on interior nodes:
\begin{equation}
L^B v = f
\end{equation}
where we have ${L} = {L}_{1-}$ and $L^B = L_{1-}^B$ with
\begin{equation}
L_{1-}^B \equiv \frac{1}{\Delta}\begin{pmatrix}
1 - (1 + \underline{\xi} \Delta) &0&0&\dots&0&0&0\\
-1&1&0&\dots&0&0&0\\
\vdots&\vdots&\vdots&\ddots&\vdots&\vdots&\vdots\\
0&0&0&\dots&-1&1&0\\
0&0&0&\cdots&0&-1&1
\end{pmatrix}_{M\times M}
\end{equation}
instead of solving the full system with boundary conditions. Similarly, one can define differential operators on the interior as follows:

\begin{align}
L_{1-}^B &\equiv \frac{1}{\Delta}\begin{pmatrix}
1 - (1 + \underline{\xi} \Delta) &0&0&\dots&0&0&0\\
-1&1&0&\dots&0&0&0\\
\vdots&\vdots&\vdots&\ddots&\vdots&\vdots&\vdots\\
0&0&0&\dots&-1&1&0\\
0&0&0&\cdots&0&-1&1
\end{pmatrix}_{M\times M}\label{eq:L-1-regular} \\
L_{1+}^B &\equiv \frac{1}{\Delta}\begin{pmatrix}
-1&1&0&\dots&0&0&0\\
0&-1&1&\dots&0&0&0\\
\vdots&\vdots&\vdots&\ddots&\vdots&\vdots&\vdots\\
0&0&0&\dots&0&-1&1\\
0&0&0&\cdots&0&0&-1+(1-\overline{\xi} \Delta)
\end{pmatrix}_{M\times M}\label{eq:L-1-plus-regular} \\
L_2^B &\equiv \frac{1}{\Delta^2}\begin{pmatrix}
-2 + (1 + \underline{\xi}\Delta) &1&0&\dots&0&0&0\\
1&-2&1&\dots&0&0&0\\
\vdots&\vdots&\vdots&\ddots&\vdots&\vdots&\vdots\\
0&0&0&\dots&1&-2&1\\
0&0&0&\cdots&0&1&-2 + (1- \overline{\xi}\Delta)
\end{pmatrix}_{M\times M}\label{eq:L-2-regular}
\end{align}

which can be defined in more compact forms:
\begin{align}
L_{1-}^B &\equiv
\text{tridiag}(-\mathbf{1}_{M-1}, \begin{bmatrix}
1-(1+\underline{\xi} \Delta) & \mathbf{1}_{M-1}^T
\end{bmatrix}^T, \mathbf{0}_{M-1} )  \\
L_{1+}^B &\equiv \text{tridiag}(\mathbf{0}_{M-1}, \begin{bmatrix}
 -\mathbf{1}_{M-1}^T & -1+(1-\overline{\xi} \Delta)
\end{bmatrix}^T, \mathbf{1}_{M-1} ) \\
L_2^B &\equiv \text{tridiag}(
\mathbf{1}_{M-1},
\begin{bmatrix}
-2+(1+\underline{\xi} \Delta)& -2\mathbf{1}_{M-2}^T & -2+(1-\overline{\xi} \Delta)
\end{bmatrix}^T,
\mathbf{1}_{M-1} )
\end{align}

\subsubsection{Absorbing Boundary Conditions}
To apply an absorbing barrier condition $\tilde{v}(x_{\min}) =S$ for some $S \in \mathbb{R}$, with one reflecting barrier condition on the upper bound $v'(x_{\max}) = 0$, one can use

\begin{equation}\label{eq:absorbing-barrier-matrix-regular}
B = \begin{bmatrix}
1 & 0 & 0 & \dots & 0 & 0 & 0 \\
0 & 0 & 0 & \dots & 0 & -1 & 1\\
\end{bmatrix}_{2 \times (M+2)} \quad
b = \begin{bmatrix}
S \\
0
\end{bmatrix}
\end{equation}

Similarly, one can apply an absorbing condition on the upper bound  $\tilde{v}(x_{\max}) =S$ for some $S \in \mathbb{R}$ and the


\begin{equation}\label{eq:absorbing-barrier-matrix-ub-regular}
B = \begin{bmatrix}
-1 & 1 & 0 & \dots & 0 & 0 & 0 \\
0 & 0 & 0 & \dots & 0 & 0 & 1\\
\end{bmatrix}_{2 \times (M+2)} \quad
b = \begin{bmatrix}
0 \\
S
\end{bmatrix}
\end{equation}


\section{Discretizing Operators with a Irregular Grid}

\subsection{Applying Boundary Conditions}
Instead of solving \eqref{eq:reflecting-barrier-extended-system} for a value function $\tilde{v}(\overline{x})$ on the extended grid, one can perform Gaussian elimination to reduce the system and solve $\tilde{v}(x)$, which gives the identical solution as the interior of $\tilde{v}(\overline{x})$.


\subsection{Irregular grids}

Define the vectors of backward and forward distance for interior nodes as follows:
\begin{align}
\Delta_{-}^\circ &= \set{\Delta_{i,-}}_{i=1}^M \\
\Delta_{+}^\circ &= \set{\Delta_{i,+}}_{i=1}^M
\end{align}


We can then define the following operators on $\overline{x}$:


\begin{equation}\label{eq:L-1-minus-extended}
{L}_{1-} \equiv\begin{pmatrix}
-\Delta_{1,-}^{-1}&\Delta_{1,-}^{-1}&0&\dots&0&0&0\\
0&-\Delta_{2,-}^{-1}&\Delta_{2,-}^{-1}&\dots&0&0&0\\
\vdots&\vdots&\vdots&\ddots&\vdots&\vdots&\vdots\\
0&0&0&\dots&\Delta_{M-1,-}^{-1}&0&0\\
0&0&0&\cdots&-\Delta_{M,-}^{-1}&\Delta_{M,-}^{-1}&0
\end{pmatrix}_{M\times (M+2)}
\end{equation}

\begin{equation}\label{eq:L-1-plus-extended}
{L}_{1+} \equiv \begin{pmatrix}
0&-\Delta_{1,+}^{-1}&\Delta_{1,+}^{-1}&\dots&0&0&0\\
0&0&-\Delta_{2,+}^{-1}&\dots&0&0&0\\
\vdots&\vdots&\vdots&\ddots&\vdots&\vdots&\vdots\\
0&0&0&\dots&-\Delta_{M-1,+}^{-1}&\Delta_{M-1,+}^{-1}&0\\
0&0&0&\cdots&0&-\Delta_{M,+}^{-1}&\Delta_{M,+}^{-1}&
\end{pmatrix}_{M\times (M+2)}
\end{equation}


\begin{equation}\label{eq:L-2-extended} \small
{L}_2 \equiv 2 \begin{psmallmatrix}
(\Delta_{1,+}+\Delta_{1,-})^{-1}\Delta_{1,-}^{-1} &-\Delta_{1,-}^{-1} \Delta_{1,+}^{-1}  & (\Delta_{1,+}+\Delta_{1,-})^{-1} \Delta_{i,+}^{-1}&\dots&0&0\\
0 & \ddots & \ddots &\ddots & 0 & 0\\
\vdots&\ddots &\ddots & \ddots & \ddots & \vdots\\
0&0&\cdots&
(\Delta_{M,+}+\Delta_{M,-})^{-1}\Delta_{M,-}^{-1} &-\Delta_{M,-}^{-1} \Delta_{M,+}^{-1}  &
(\Delta_{M,+}+\Delta_{M,-})^{-1}\Delta_{M,+}^{-1}
\end{psmallmatrix}_{M\times (M+2)}
\end{equation}

Note that we use the following discretization scheme scheme from \cite{achdou17}:
\begin{equation}
v''(x_i) \approx \dfrac{ \Delta_{i,-} \tilde{v}( x_i + \Delta_{i,+}) - (\Delta_{i,+} + \Delta_{i,-}) \tilde{v}( x_i ) + \Delta_{i,+} \tilde{v}( x_i - \Delta_{i,-})}{\frac{1}{2}(\Delta_{i,+} + \Delta_{i,-}) \Delta_{i,+} \Delta_{i,-} }, \text{for } i = 1, \ldots, M
\end{equation}
for second-order derivatives.

And one for identity matrix:
\begin{equation}\label{eq:I-extended}
{I} \equiv \begin{pmatrix}
0 & 1 & 0 & \cdots & 0 & 0 & 0\\
0 & 0 & 1 & \cdots & 0 & 0 & 0\\
\vdots & \vdots  & \ddots & \ddots &  \vdots  & \vdots  & \vdots  \\
0 & 0 & 0 & \cdots & 1 & 0 & 0 \\
0 & 0 & 0 & \cdots & 0 & 1 & 0
\end{pmatrix}_{M\times (M+2)}
\end{equation}

Alternatively, in a more compact form, using vectors distances for interior nodes:
\begin{align}
{L}_{1-} \equiv
\begin{bmatrix}
\text{tridiag($\mathbf{0}_{M-1}$, $-(\Delta_-^\circ)^{-1}$, $(\Delta_-^\circ)^{-1}$) } &  \mathbf{0}_M
\end{bmatrix}
\end{align}
\begin{align}
{L}_{1+} \equiv
\begin{bmatrix}
\mathbf{0}_M &
\text{tridiag($\mathbf{0}_{M-1}$, $-(\Delta_-^\circ)^{-1}$, $(\Delta_-^\circ)^{-1}$) }
\end{bmatrix}
\end{align}
\begin{align}
{L}_{2} \equiv
2 \odot \text{tridiag}^+ \left[(\Delta_-^\circ + \Delta_+^\circ)^{-1} \odot (\Delta_{-}^\circ)^{-1}, -(\Delta_-^\circ \odot \Delta_+^\circ)^{-1} , (\Delta_-^\circ + \Delta_+^\circ)^{-1} \odot (\Delta_{+}^\circ)^{-1} \right]
\end{align}
\begin{align}
{I} \equiv
\begin{bmatrix}
\mathbf{0}_M &  \text{diag($\mathbf{1}_M$)} &  \mathbf{0}_M
\end{bmatrix}
\end{align}

\subsubsection{Mixed boundary conditions}
Recall mixed boundary conditions from \eqref{eq:mixed-BC1} and \eqref{eq:mixed-BC2}. Note that reflecting barrier conditions are special cases with $\overline{\xi} = \underline{\xi} = 0$. Using forward difference and backward difference discretization scheme for the lower bound and upper bound respectively, we have
\begin{align}
&\frac{\overline{v}_1 - \overline{v}_0}{\Delta_{0,+}} - \underline{\xi} \overline{v}_0 = 0 \\
&\frac{\overline{v}_{M+1} - \overline{v}_M}{\Delta_{M+1,-}} - \overline{\xi} \overline{v}_{M+1} = 0
\end{align}

Thus, the corresponding boundary condition matrix $B$ is

\begin{equation}\label{eq:mixed-barrier-matrix-original}
B = \begin{bmatrix}
-\dfrac{1}{\Delta_{0,+}} + \underline{\xi} & \dfrac{1}{\Delta_{0,+}} & 0 & \dots & 0 & 0 & 0 \\
0 & 0 & 0 & \dots & 0 & -\dfrac{1}{\Delta_{M+1,-}} & \dfrac{1}{\Delta_{M+1,-}} + \overline{\xi}\\
\end{bmatrix}_{2 \times (M+2)} \quad
b = \begin{bmatrix}
0 \\
0
\end{bmatrix}
\end{equation}
which provides the identical system as
\begin{equation}\label{eq:mixed-barrier-matrix}
B = \begin{bmatrix}
-1 +  \underline{\xi} \Delta_{1,-} & 1 & 0 & \dots & 0 & 0 & 0 \\
0 & 0 & 0 & \dots & 0 & -1 & 1 + \overline{\xi} \Delta_{M,+}\\
\end{bmatrix}_{2 \times (M+2)} \quad
b = \begin{bmatrix}
0 \\
0
\end{bmatrix}
\end{equation}
since $\Delta_{0,+} = \Delta_{1,-}$ and $\Delta_{M+1,-} = \Delta_{M,+}$


The first columns of all the extension operators above, $\overline{L}_{1-}, \overline{L}_{1+}, \overline{L}_{2}, \overline{I}$, have non-zero element only in the first rows. Thus, a single Gaussian elimination for the first extension grid will suffice to remove the extended. Likewise, in the last columns of all the extension operators have non-zero element only in the last row.


Subtracting the first row of $\overline{L}_{1-}$ by the first row of $B$ in \eqref{eq:mixed-barrier-matrix} multiplied by $(-1 + \underline{\xi} \Delta_{1,-} )^{-1}\Delta_{1,-}^{-1}$ gives, with the corresponding row operation matrix $R$ for Gaussian elimination,

\begin{equation}
R {L}_{1-} = \begin{pmatrix}
0&\Delta_{1,-}^{-1} \left[ 1 - (1+\underline{\xi} \Delta_{1,-} )\right] &0&\dots&0&0&0\\
0&-\Delta_{2,-}^{-1}&\Delta_{2,-}^{-1}&\dots&0&0&0\\
\vdots&\vdots&\vdots&\ddots&\vdots&\vdots&\vdots\\
0&0&0&\dots&\Delta_{M-1,-}^{-1}&0&0\\
0&0&0&\cdots&-\Delta_{M,-}^{-1}&\Delta_{M,-}^{-1}&0
\end{pmatrix}_{M\times (M+2)}
\end{equation}
note that there is no zero element in the first and last column for nodes on boundaries. Hence, solving the corresponding extended system,
\begin{align}
\begin{bmatrix}
{L}\\
B
\end{bmatrix} =
\begin{bmatrix}
f \\ b
\end{bmatrix}
\end{align}
is identical as solving the following system
\begin{align}
R
\begin{bmatrix}
{L}\\
B
\end{bmatrix} &=
R
\begin{bmatrix}
f \\ b
\end{bmatrix} \\
&= \begin{bmatrix}
f \\ b
\end{bmatrix}
\end{align}
as $b$ is a zero vector so that the row operations $R$ do not change anything on the RHS. Furthermore, limited to the interior, solving $v$ in the system above is identical as solving the following system with an operator $L^B$ on interior nodes:
\begin{equation}
L^B v = f
\end{equation}
where we have ${L} = {L}_{1-}$ and $L^B = L_{1-}^B$ with
\begin{equation}
L_{1-}^B \equiv \begin{pmatrix}
\Delta_{1,-}^{-1} [1 - (1 + \underline{\xi} \Delta_{1,-})] &0&0&\dots&0&0&0\\
-\Delta_{2,-}^{-1}&\Delta_{2,-}^{-1}&0&\dots&0&0&0\\
\vdots&\vdots&\vdots&\ddots&\vdots&\vdots&\vdots\\
0&0&0&\dots&-\Delta_{M-1,-}^{-1}&\Delta_{M-1,-}^{-1}&0\\
0&0&0&\cdots&0&-\Delta_{M,-}^{-1}&\Delta_{M,-}^{-1}
\end{pmatrix}_{M\times M}
\end{equation}
instead of solving the full system with boundary conditions. Similarly, one can define differential operators on the interior as follows:


\begin{align}
L_{1-}^B &\equiv \begin{pmatrix}
\Delta^{-1}_{1,-} [1 - (1 + \underline{\xi} \Delta_{1,-})] &0&0&\dots&0&0&0\\
-\Delta_{2,-}^{-1}&\Delta_{2,-}^{-1}&0&\dots&0&0&0\\
\vdots&\vdots&\vdots&\ddots&\vdots&\vdots&\vdots\\
0&0&0&\dots&-\Delta_{M-1,-}^{-1}&\Delta_{M-1,-}^{-1}&0\\
0&0&0&\cdots&0&-\Delta_{M,-}^{-1}&\Delta_{M,-}^{-1}
\end{pmatrix}_{M\times M}\label{eq:L-1} \\
L_{1+}^B &\equiv \begin{pmatrix}
-\Delta_{1,+}^{-1}&\Delta_{1,+}^{-1}&0&\dots&0&0&0\\
0&-\Delta_{2,+}^{-1}&\Delta_{2,+}^{-1}&\dots&0&0&0\\
\vdots&\vdots&\vdots&\ddots&\vdots&\vdots&\vdots\\
0&0&0&\cdots&0&-\Delta_{M-1,+}^{-1}&\Delta_{M-1,+}^{-1}\\
0&0&0&\dots&0&0&\Delta_{M,+}^{-1}  [-1 + (1 - \overline{\xi} \Delta_{M,+})]  \\
\end{pmatrix}_{M\times M}\label{eq:L-1-plus} \\
L_2^B &\equiv 2 \begin{psmallmatrix}
\Xi_1 &
(\Delta_{1,+}+\Delta_{1,-})^{-1} \Delta_{1,+}^{-1}
&0&\cdots&0&0&0 \\
\vdots&\ddots&\ddots&\ddots&\ddots&\vdots&\vdots\\
0&\cdots&
(\Delta_{i,+}+\Delta_{i,-})^{-1} \Delta_{i,-}^{-1} &
-\Delta_{i,-}^{-1} \Delta_{i,+}^{-1}  &
 (\Delta_{i,+}+\Delta_{i,-})^{-1} \Delta_{i,+}^{-1}&\cdots&0 \\
\vdots&\vdots&\vdots&\ddots&\ddots&\ddots&\vdots\\
0&0&0&\cdots&0&(\Delta_{M,+}+\Delta_{M,-})^{-1} \Delta_{M,-}^{-1}&\Xi_M
\end{psmallmatrix}_{M\times M}\label{eq:L-2} \\
I^B &\equiv \begin{pmatrix}
1 & 0 & \cdots & 0 & 0 \\
0 & 1 & \cdots & 0 & 0 \\
\vdots  & \ddots & \ddots &  \vdots  & \vdots   \\
0 & 0 & \cdots & 1 & 0 \\
0 & 0 & \cdots & 0 & 1
\end{pmatrix}_{M\times M}
\end{align}
where

\begin{align}
\Xi_{1} &= -2\left[  \Delta_{1,-}^{-1} \Delta_{1,+}^{-1} + (-1 + \underline{\xi} \Delta_{1,-})^{-1} (\Delta_{1,+} + \Delta_{1,-})^{-1}  \Delta_{1,-}^{-1} \right] \\
\Xi_{M} &= -2\left[  \Delta_{M,-}^{-1} \Delta_{M,+}^{-1} - (1 + \overline{\xi} \Delta_{M,+})^{-1} (\Delta_{M,+} + \Delta_{M,-})^{-1}  \Delta_{M,+}^{-1} \right]
\end{align}


Or, alternatively, with vectorized differences:


\begin{align}
{L}_{1-}^B \equiv
\text{tridiag} \left[-(\Delta_-^\circ)^{-1}[2:M], \begin{pmatrix}\Delta^{-1}_{1,-} [1 - (1 + \underline{\xi} \Delta_{1,-})]; (\Delta_-^\circ )^{-1}[2:M] \end{pmatrix}, \mathbf{0}_{M-1}  \right]
\end{align}

\begin{align}
{L}_{1+}^B \equiv
\text{tridiag} \left[  \mathbf{0}_{M-1},
\begin{pmatrix}-(\Delta_-^\circ )^{-1}[1:M-1]; \Delta^{-1}_{M,+} [-1 + (1 - \overline{\xi} \Delta_{M,+})] \end{pmatrix},
(\Delta_-^\circ)^{-1}[2:M]
 \right]
\end{align}
\begin{align}
{L}_{2}^B \equiv
2 \odot \text{tridiag}& \Big[(\Delta_-^\circ + \Delta_+^\circ)^{-1} \odot (\Delta_{-}^\circ)^{-1}, \\&
	\begin{pmatrix}
	\Xi_1 ;
	-(\Delta_-^\circ \odot \Delta_+^\circ)^{-1}[2:M-1];
	\Xi_M
	\end{pmatrix} , \\&
	(\Delta_-^\circ + \Delta_+^\circ)^{-1} \odot (\Delta_{+}^\circ)^{-1} \Big]
\end{align}

\begin{align}
I^B \equiv  \text{diag($\mathbf{1}_M$)}
\end{align}


\subsubsection{Absorbing barrier conditions}

To apply an absorbing barrier condition $\tilde{v}(x_{\min}) =S$ for some $S \in \mathbb{R}$, with one reflecting barrier condition on the upper bound $v'(x_{\max}) = 0$, one can use

\begin{equation}\label{eq:absorbing-barrier-matrix}
B = \begin{bmatrix}
1 & 0 & 0 & \dots & 0 & 0 & 0 \\
0 & 0 & 0 & \dots & 0 & -1 & 1\\
\end{bmatrix}_{2 \times (M+2)} \quad
b = \begin{bmatrix}
S \\
0
\end{bmatrix}
\end{equation}

Similarly, one can apply an absorbing condition on the upper bound  $\tilde{v}(x_{\max}) =S$ for some $S \in \mathbb{R}$ and the


\begin{equation}\label{eq:absorbing-barrier-matrix-ub}
B = \begin{bmatrix}
-1 & 1 & 0 & \dots & 0 & 0 & 0 \\
0 & 0 & 0 & \dots & 0 & 0 & 1\\
\end{bmatrix}_{2 \times (M+2)} \quad
b = \begin{bmatrix}
0 \\
S
\end{bmatrix}
\end{equation}

Note that the corresponding $B$ and $b$ are identical with regular grid cases.

\subsection{Examples}

\begin{example}\label{ex:gaussian-elimination-reflecting-barrier}
	Consider $L = L_{2}$ to solve $L v = f$ with $M = 3$ under uniform grids $\overline{x} = \{x_0, x_1, x_2, x_3, x_4\}$ and $\Delta = 1$, whose corresponding interior grid is $x = \{x_1, x_2, x_3\}$. This gives
	\begin{align}
	L^B = 	 \begin{bmatrix}
	-2 & 1 & 0 \\
	1 & -2 & 1 \\
	0 & 1 & -2 \\
	\end{bmatrix}
	\end{align}
	so $L^B v= f$ on the grid $x$ results in the following system
	\begin{align} \label{eq:extended-system-reflecting-barrier-reduced-system}
	\begin{bmatrix}
	-1 & 1 & 0  \\
	1 & -2 & 1 \\
	0 & 1 & -1 \\
	\end{bmatrix} 	  \begin{bmatrix}
	\tilde{v}(x_1) \\
	\tilde{v}(x_2) \\
	\tilde{v}(x_3)
	\end{bmatrix}
	=
	\begin{bmatrix}
	\tilde{f}(x_1) \\
	\tilde{f}(x_2) \\
	\tilde{f}(x_3)
	\end{bmatrix}
	\end{align}


	For the extended system we have
	\begin{align}
	{L} &=
	\begin{bmatrix}
	1 & -2 & 1 & 0 & 0 \\
	0 & 1 & -2 & 1 & 0 \\
	0 & 0 & 1 & -2 & 1 \\
	\end{bmatrix} \\
	B &= \begin{bmatrix}
	-1 & 1  & 0 & 0 & 0 \\
	0 & 0 & 0 & -1 & 1
	\end{bmatrix}
	\end{align}

	Constructing the stacked extended system \eqref{eq:reflecting-barrier-extended-system} gives
	\begin{align}\label{eq:extended-system-reflecting-barrier-before-gaussian-elimination}
	\begin{bmatrix}
	1 & -2 & 1 & 0 & 0 \\
	0 & 1 & -2 & 1 & 0 \\
	0 & 0 & 1 & -2 & 1 \\
	-1 & 1  & 0 & 0 & 0 \\
	0 & 0 & 0 & -1 & 1
	\end{bmatrix}
	\begin{bmatrix}
	\tilde{v}(x_0) \\
	\tilde{v}(x_1) \\
	\tilde{v}(x_2) \\
	\tilde{v}(x_3) \\
	\tilde{v}(x_4)
	\end{bmatrix}
	=
	\begin{bmatrix}
	\tilde{f}(x_1) \\
	\tilde{f}(x_2) \\
	\tilde{f}(x_3) \\
	0 \\
	0
	\end{bmatrix}
	\end{align}

	Note that substracting the first row of ${L}$ by $(-1)$ times the first row of $B$ returns an identical system as \eqref{eq:extended-system-reflecting-barrier-before-gaussian-elimination}. Likewise, substracting the last row of ${L}$ by $(-1)$ times the last row of $B$ returns the identical system. Performing the two Gaussian elimination yields the following system:
	\begin{align}\label{eq:extended-system-reflecting-barrier-after-gaussian-elimination}
	\begin{bmatrix}
	0 & -1 & 1 & 0 & 0 \\
	0 & 1 & -2 & 1 & 0 \\
	0 & 0 & 1 & -1 & 0 \\
	1 & -1  & 0 & 0 & 0 \\
	0 & 0 & 0 & -1 & 1
	\end{bmatrix} 	  \begin{bmatrix}
	\tilde{v}(x_0) \\
	\tilde{v}(x_1) \\
	\tilde{v}(x_2) \\
	\tilde{v}(x_3) \\
	\tilde{v}(x_4)  \\
	\end{bmatrix}
	=
	\begin{bmatrix}
	\tilde{f}(x_2) \\
	\tilde{f}(x_3) \\
	\tilde{f}(x_4) \\
	0 \\
	0
	\end{bmatrix}
	\end{align}
	Note that now we have the first three rows of the coefficient matrix with zero columns on the extended nodes, $\tilde{v}(x_1 - \Delta)$ and $\tilde{v}(x_3 + \Delta)$. Extracting the system corresponding to the first three rows returns the following system, which solves the interior of $\bar{v}$, i.e., $v$:
	\begin{align}
	\begin{bmatrix}
	-1 & 1 & 0  \\
	1 & -2 & 1 \\
	0 & 1 & -1 \\
	\end{bmatrix} 	  \begin{bmatrix}
	\tilde{v}(x_1) \\
	\tilde{v}(x_2) \\
	\tilde{v}(x_3)
	\end{bmatrix}
	=
	\begin{bmatrix}
	\tilde{f}(x_1) \\
	\tilde{f}(x_2) \\
	\tilde{f}(x_3)
	\end{bmatrix}
	\end{align}
	which is identical as  \eqref{eq:extended-system-reflecting-barrier-reduced-system}.
\end{example}

\begin{example}
	Consider solving \eqref{eq:extended-system-reflecting-barrier-reduced-system}, but this time with an absorbing barrier condition on the lower bound, $\tilde{v}(x_{\min}) = S$ with a boundary condition matrix
	\begin{align}
	B &= \begin{bmatrix}
	1 & 0  & 0 & 0 & 0 \\
	0 & 0 & 0 & -1 & 1
	\end{bmatrix}
	\end{align}
	The corresponding extended system is
	\begin{align}\label{eq:extended-system-absorbing-barrier-before-gaussian-elimination}
	\begin{bmatrix}
	1 & -2 & 1 & 0 & 0 \\
	0 & 1 & -2 & 1 & 0 \\
	0 & 0 & 1 & -2 & 1 \\
	1 & 0  & 0 & 0 & 0 \\
	0 & 0 & 0 & -1 & 1
	\end{bmatrix}
	\begin{bmatrix}
	\tilde{v}(x_0) \\
	\tilde{v}(x_1) \\
	\tilde{v}(x_2) \\
	\tilde{v}(x_3) \\
	\tilde{v}(x_4) \\
	\end{bmatrix}
	=
	\begin{bmatrix}
	\tilde{f}(x_1) \\
	\tilde{f}(x_2) \\
	\tilde{f}(x_3) \\
	S \\
	0
	\end{bmatrix}
	\end{align}

	Note that substracting the first row of ${L}$ by $(-1)$ times the first row of $B$ returns an identical system as \eqref{eq:extended-system-absorbing-barrier-before-gaussian-elimination}. Likewise, substracting the last row of ${L}$ by $(-1)$ times the last row of $B$ returns the identical system. Performing the two Gaussian elimination yields the following system:
	\begin{align}\label{eq:extended-system-absorbing-barrier-after-gaussian-elimination}
	\begin{bmatrix}
	0 & -2 & 1 & 0 & 0 \\
	0 & 1 & -2 & 1 & 0 \\
	0 & 0 & 1 & -1 & 0 \\
	1 & 0  & 0 & 0 & 0 \\
	0 & 0 & 0 & -1 & 1
	\end{bmatrix} 	  \begin{bmatrix}
	\tilde{v}(x_0) \\
	\tilde{v}(x_1) \\
	\tilde{v}(x_2) \\
	\tilde{v}(x_3) \\
	\tilde{v}(x_4) \\
	\end{bmatrix}
	=
	\begin{bmatrix}
	\tilde{f}(x_1) - S \\
	\tilde{f}(x_2) \\
	\tilde{f}(x_3) \\
	0 \\
	0
	\end{bmatrix}
	\end{align}

	Extracting the system corresponding to the first three rows returns the following system, which solves the interior of $\bar{v}$, i.e., $v$:
	\begin{align}
	\begin{bmatrix}
	-2 & 1 & 0  \\
	1 & -2 & 1 \\
	0 & 1 & -1 \\
	\end{bmatrix} 	  \begin{bmatrix}
	\tilde{v}(x_1) \\
	\tilde{v}(x_2) \\
	\tilde{v}(x_3)
	\end{bmatrix}
	=
	\begin{bmatrix}
	\tilde{f}(x_1) - S \\
	\tilde{f}(x_2) \\
	\tilde{f}(x_3)
	\end{bmatrix}
	\end{align}
\end{example}

\section{Applications}

\subsection{Hamilton–Jacobi–Bellman equations (HJBE)}
Consider solving for $v$ from the following optimal control problem
\begin{align}
\tilde{v}(x_0) = \max_{ {\{\alpha(t) \} }_{t \geq 0} } \int_{0}^\infty e^{-\rho t} r( x(t), \alpha(t )) dt
\end{align}

with the law of motion for the state
\begin{align}
dx = \mu dt + \sigma dW
\end{align}


for some constant $\mu \geq 0$ and $\sigma \geq 0$ with $x(0) = x_0$.

Let $\alpha^*(t)$ be the optimal solution. Suppose that $r$ under $\alpha^*(t)$ can be expressed in terms of state variables, $r^* (x)$. Then, the HJBE yields

\begin{equation}\label{eq:hamilton-jacobi-bellman}
\rho \tilde{v}(x) = r^*(x) +  \mu  \partial_{x} \tilde{v}(x) + \dfrac{\sigma^2}{2} \partial_{xx} \tilde{v}(x)
\end{equation}

In terms of differential operators, one can rewrite the equation as
\begin{equation}\label{eq:hjbe-system-function}
\tilde{L}  \tilde{v}(x) = r^*(x)
\end{equation}

with $\tilde{L} = \rho -  \tilde{L}_x$ where $\tilde{L}_x$ is defined as

\begin{equation}\label{eq:L-defn}
\tilde{L}_x = \mu \partial_{x} + (\sigma^2/2) \partial_{xx}
\end{equation}


By descretizing the space of $x$, one can solve the corresponding system by using discretized operators for $\partial_{x}$ ($L_{1+}$), $\partial_{xx}$ ($L_2$) on some grids of length $M$, $\{x_i\}_{i=1}^M$:
\begin{align}\label{eq:L-descritized-defn}
L_x = \mu L_{1+} + \dfrac{\sigma^2}{2} L_{2}
\end{align}
so that $v$ from \eqref{eq:hjbe-system-function} can be computed by solving the following discretized system of equations:
\begin{align}
L v &= r^*
\end{align}
where $v$ and $r^*$ are $M$-vectors whose $i$th elements are $\tilde{v}(x_i)$ and $r^*(x_i)$, respectively, and $L$ is defined as $L = \rho I - L_x$.



\subsection{Kolmogorov forward equations (KFE) under diffusion process}
Let $g(x, t)$ be the distribution of $x$ at time $t$ from the example above. By the Kolmogorov forward equation, the following PDE holds:

\begin{equation}\label{eq:kfe}
\partial_{t} g(x, t) = - \mu \partial_{x}  g(x,t) + \dfrac{\sigma^2}{2} \partial_{xx} g(x,t)
\end{equation}

\subsubsection{Stationary distributions}
The stationary distribution $g^*(x)$ satisfies

\begin{equation}\label{eq:kfe-stationary}
0 = - \mu \partial_{x} g^*(x) + \dfrac{\sigma^2}{2} \partial_{xx} g^*(x)
\end{equation}

which can be rewritten as

\begin{equation}
\tilde{L}^*_x g(x) = 0
\end{equation}

where

\begin{equation}\label{eq:L_star-defn}
\tilde{L}^*_x =  - \mu \partial_{x} + (\sigma^2/2) \partial_{xx}
\end{equation}

By descretizing the space of $x$, one can solve the corresponding system by using discretized operators for $\tilde{L}^*_x$. Note that the operator for the KFE in \eqref{eq:L_star-defn} is the adjoint operator of the operator for the HJBE in \eqref{eq:L-defn}, and the correct discretization scheme for \eqref{eq:L_star-defn} can be, analogously, done by taking the transpose of the discretized operator for HJBE in \eqref{eq:L-descritized-defn} -- see \cite{gabaix16} and \cite{achdou17}. Hence, one can find the stationary distribution by solving the following discretized system of equations:

\begin{equation}
L^T_x g = 0
\end{equation}
where $L^T_x$ is the transpose of $L_x$ from \eqref{eq:L-descritized-defn} and $g$ is an $M$-vector whose element is $g(x_i)$ such that $\sum_{i=1}^M g(x_i) = 1$.

\subsubsection{Full dynamics of distributions}
One can also solve the full PDE in \eqref{eq:kfe}, given an initial distribution $g(x, 0)$. After discretization, note that \eqref{eq:kfe} can be rewritten as
\begin{equation}\label{eq:kfe-discretized}
\dot{g}(t) = L^T_x g(t)
\end{equation}
where $\dot{g}(t)$ is an $M$-vector whose $i$th element is $\partial_{t} g(x_i, t)$, which can be efficently solved by a number of differential equation solvers available in public, including \cite{rackauckas17}.

\newpage

\appendix
\section{Derivation by substitution}
One can also derive the formula for $L_{1-}^B, L_{1+}^B, L_2^B$ in \eqref{eq:L-1-regular}, \eqref{eq:L-1-plus-regular}, \eqref{eq:L-2-regular} by substitution. For simplicity, here we focus on the case when we have regular grids, i.e., $x_{i+1} - x_i = \Delta$ for some $\Delta > 0$ for all $i = 0, ..., M$.

Using the backwards first-order difference, \eqref{eq:mixed-BC1} implies
\begin{align}
\dfrac{\tilde{v}({x_{1}}) - \tilde{v}(x_{0})}{\Delta} &= - \underline{\xi} \tilde{v}({x_{0}})
\end{align}
i.e.,
\begin{align}\label{eq:BC1-extrapolation-uniform}
\tilde{v}(x_0) = \frac{1}{1-\underline{\xi} \Delta } \tilde{v}(x_1)
\end{align}
at the lower bound.

Likewise, \eqref{eq:mixed-BC2} under the forwards first-order difference yields
\begin{align}
\dfrac{\tilde{v}(x_{M+1}) - \tilde{v}({x_{M}})}{\Delta} &= - \overline{\xi} \tilde{v}({x_{M+1} })
\end{align}
i.e.,
\begin{align}\label{eq:BC2-extrapolation-uniform}
\tilde{v}(x_{M+1}) = \frac{1}{1+\overline{\xi} \Delta } \tilde{v}(x_M)
\end{align}
at the upper bound.

The discretized central difference of second order under \eqref{eq:mixed-BC1} at the lower bound is, by substituting \eqref{eq:BC1-extrapolation-uniform} in,
\begin{align}
\dfrac{\tilde{v}({x_{1}} + \Delta) - 2 \tilde{v}({x_{1}}) + \tilde{v}(x_{\min})}{\Delta^2} &=   \dfrac{\tilde{v}({x_{1}} + \Delta) - \tilde{v}({x_{1}})}{\Delta^2} - \dfrac{1}{\Delta}\dfrac{\tilde{v}({x_{1}}) - \tilde{v}(x_{\min}) }{\Delta}  \\
&= \dfrac{\tilde{v}({x_{1}} + \Delta) - \tilde{v}({x_{1}})}{\Delta^2} + \dfrac{1}{\Delta} \underline{\xi} \tilde{v}({x_{1}})  \\
&= \dfrac{1}{\Delta^2}  (- 1 + \Delta \underline{\xi}) \tilde{v}({x_{1}})  + \dfrac{1}{\Delta^2}  \tilde{v}({x_{1}} + \Delta)
\end{align}
Similarly, by \eqref{eq:mixed-BC2}, we have
\begin{align}
\dfrac{\tilde{v}(x_{\max}) - 2 \tilde{v}({x_{M}} ) + \tilde{v}({x_{M}} -\Delta)}{\Delta^2} &=   \dfrac{\tilde{v}({x_{M}} - \Delta) - \tilde{v}({x_{M}})}{\Delta^2} + \dfrac{1}{\Delta}\dfrac{ \tilde{v}(x_{\max}) - \tilde{v}({x_{M}}) }{\Delta}  \\
&= \dfrac{\tilde{v}({x_{M}} - \Delta) - \tilde{v}({x_{M}})}{\Delta^2}  - \dfrac{1}{\Delta} \overline{\xi} \tilde{v}({x_{M}})  \\
&= \dfrac{1}{\Delta^2}  (- 1 - \Delta \overline{\xi}) \tilde{v}({x_{M}})  + \dfrac{1}{\Delta^2}  \tilde{v}({x_{M}} - \Delta)
\end{align}
at the upper bound.

Thus, the corresponding discretized differential operator $L_{1-}$, $L_{1+}$, and $L_2$ are defined as

\begin{align}
L_{1-}^B &\equiv \frac{1}{\Delta}\begin{pmatrix}
1 - (1 + \underline{\xi} \Delta) &0&0&\dots&0&0&0\\
-1&1&0&\dots&0&0&0\\
\vdots&\vdots&\vdots&\ddots&\vdots&\vdots&\vdots\\
0&0&0&\dots&-1&1&0\\
0&0&0&\cdots&0&-1&1
\end{pmatrix}_{M\times M} \\
L_{1+}^B &\equiv \frac{1}{\Delta}\begin{pmatrix}
-1&1&0&\dots&0&0&0\\
0&-1&1&\dots&0&0&0\\
\vdots&\vdots&\vdots&\ddots&\vdots&\vdots&\vdots\\
0&0&0&\dots&0&-1&1\\
0&0&0&\cdots&0&0&-1+(1-\overline{\xi} \Delta)
\end{pmatrix}_{M\times M} \\
L_2^B &\equiv \frac{1}{\Delta^2}\begin{pmatrix}
-2 + (1 + \underline{\xi}\Delta) &1&0&\dots&0&0&0\\
1&-2&1&\dots&0&0&0\\
\vdots&\vdots&\vdots&\ddots&\vdots&\vdots&\vdots\\
0&0&0&\dots&1&-2&1\\
0&0&0&\cdots&0&1&-2 + (1- \overline{\xi}\Delta)
\end{pmatrix}_{M\times M}
\end{align}


\iffalse
\subsection{Differential operators by basis}
Define the following basis matrices:

\begin{align}
U_1^{-} &\equiv \begin{pmatrix}
1  &0&0&\dots&0&0&0\\
-1&1&0&\dots&0&0&0\\
\vdots&\vdots&\vdots&\ddots&\vdots&\vdots&\vdots\\
0&0&0&\dots&-1&1&0\\
0&0&0&\cdots&0&-1&1
\end{pmatrix}_{M\times M}\label{eq:L-1-basis} \\
U_1^{+} &\equiv \begin{pmatrix}
-1  &1&0&\dots&0&0&0\\
0&-1&1&\dots&0&0&0\\
\vdots&\vdots&\vdots&\ddots&\vdots&\vdots&\vdots\\
0&0&0&\dots&0&-1&1\\
0&0&0&\cdots&0&0&-1
\end{pmatrix}_{M\times M}\label{eq:L-1+-basis} \\
\end{align}

and the boundary conditions for the reflecting conditions:

\begin{align}
B_{1}  &\equiv \begin{pmatrix}
(1 + \underline{\xi} \Delta^{-1}_{1,-}) &0&\dots&0&0\\
0&0&\dots&0&0\\
\vdots&\vdots&\ddots&\vdots&\vdots\\
0&0&\cdots&0&0\\
0&0&\cdots&0&0
\end{pmatrix}_{M\times M} \\
B_{M}  &\equiv \begin{pmatrix}
0 &0&\dots&0&0\\
0&0&\dots&0&0\\
\vdots&\vdots&\ddots&\vdots&\vdots\\
0&0&\cdots&0&0\\
0&0&\cdots&0&(1 - \overline{\xi} \Delta^{-1}_{M,+})
\end{pmatrix}_{M\times M}
\end{align}

\subsubsection{Regular grids}
For regular grids with the uniform distance of $\Delta > 0$, \eqref{eq:L-1-regular} and \eqref{eq:L-2-regular} can be represented by

\begin{align}
L_{1-} &= \dfrac{1}{\Delta} U_1^{-} - B_1 \\
L_{1+} &= \dfrac{1}{\Delta} U_1^{+} + B_{M} \\
L_2 &= \dfrac{1}{\Delta^2} (U_1^+ - U_1^-) + B_1 + B_{M}
\end{align}

\subsubsection{Irregular grids}
For notational brevity, for vectors with the same size, $x_1, x_2$, define $x_1 x_2$ as the elementwise-multiplied vector. Also, let $\Delta_-^\circ, \Delta_+^\circ$ as the vectors of differences on the interior nodes, i.e., $\Delta_{-}^\circ = \set{\Delta_{i,-}}_{i=1}^M$, $\Delta_{+}^\circ = \set{\Delta_{i,}}_{i=1}^M$. Then, we have
\begin{align}
L_{1-} &= \text{diag}(\Delta_{-}^{\circ} )^{-1} U_1^{-} - B_1 \\
L_{1+} &= \text{diag}(\Delta_{+}^{\circ} )^{-1} U_1^{+} + B_{M} \\
L_2 &= \text{diag} \left[ \frac{1}{2} ( {\Delta_{+}^{\circ}} + {\Delta_{-}^{\circ}} ) {\Delta_{+}^{\circ}} \right]^{-1}  U_1^{+} -
\text{diag} \left[ \frac{1}{2} ( {\Delta_{+}^{\circ}} + {\Delta_{-}^{\circ}} ) {\Delta_{-}^{\circ}} \right]^{-1}  U_1^{-}
+ B_1 + B_{M}
\end{align}
We can simplify this expression further by introducing a new notation. Let $x^{-1}$ be defined as the elementwise inverse of a vector $x$ that contains no zero element. Then, $L_2$ can be represented as
\begin{align}
L_2 &=
2\left[ \text{diag} \left( ( {\Delta_{+}^{\circ}} + {\Delta_{-}^{\circ}} )^{-1} {\Delta_{+}^{\circ}}^{-1} \right) U_1^{+} -
\text{diag} \left( ( {\Delta_{+}^{\circ}} + {\Delta_{-}^{\circ}} )^{-1} {\Delta_{-}^{\circ}}^{-1} \right) U_1^{-}  \right]
+ B_1 + B_{M} \\ \label{eq:L-2-by-basis}
&= 2 \text{diag} \left( ( {\Delta_{+}^{\circ}} + {\Delta_{-}^{\circ}} )^{-1} \right) \left[ \text{diag} \left(  {\Delta_{+}^{\circ}}^{-1} \right) U_1^{+} -
\text{diag} \left(  {\Delta_{-}^{\circ}}^{-1} \right) U_1^{-}  \right]
+ B_1 + B_{M}
\end{align}


The diagonal elements of \eqref{eq:L-2-by-basis} are also identical with the one provided in \eqref{eq:L-2} -- to see this, note that the diagonal elements of \eqref{eq:L-2-by-basis}, modulo $B_1$ and $B_{M}$, are
\begin{align}
-2 \left[ ({\Delta_{+}^{\circ}} + {\Delta_{-}^{\circ}})^{-1} {\Delta_{+}^{\circ}}^{-1} + ({\Delta_{+}^{\circ}} + {\Delta_{-}^{\circ}})^{-1} {\Delta_{-}^{\circ}}^{-1} \right] &= -2 ({\Delta_{+}^{\circ}} + {\Delta_{-}^{\circ}})^{-1}  ( {\Delta_{+}^{\circ}}^{-1} + {\Delta_{-}^{\circ}}^{-1} ) \\
&= -2({\Delta_{+}^{\circ}} + {\Delta_{-}^{\circ}})^{-1} ({\Delta_{+}^{\circ}}^{-1} {\Delta_{-}^{\circ}}^{-1}) ({\Delta_{+}^{\circ}} + {\Delta_{-}^{\circ}} )  \\
&= -2 ({\Delta_{+}^{\circ}}^{-1} {\Delta_{-}^{\circ}}^{-1})
\end{align}
which is identical with $\text{diag} (L_2)$ with $L_2$ from \eqref{eq:L-2} except the first row and last row that are affected by $B_1$ and $B_{M}$.

\fi


\bibliography{hact}
% The notebook is based on my previous version that can be found at https://github.com/ubcecon/computing_and_datascience/blob/a33d14191afba2d13598b331b3623a9512dccc90/continuous_time_methods/notes/discretized-differential-operator-derivation.tex by @chiyahn
\end{document}
