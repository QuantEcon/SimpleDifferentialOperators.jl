% !TEX program = pdflatex

\documentclass[11pt]{article}
\usepackage{amsmath,amsfonts,amsthm,amssymb,geometry,dsfont}
\usepackage[usenames,dvipsnames,svgnamesable]{xcolor}
\usepackage[capitalise,noabbrev]{cleveref} %
\usepackage{natbib}
\crefname{equation}{}{} %
\crefname{assumption}{Assumption}{Assumptions}
\crefname{property}{Property}{Properties}
\geometry{left=1in,right=1in,top=0.6in,bottom=1in}

\newcommand{\D}[1][]{\ensuremath{\boldsymbol{\partial}_{#1}}}
\newcommand{\R}{\ensuremath{\mathbb{R}}}
\newcommand{\diff}{\ensuremath{\mathrm{d}}}
\newcommand{\set}[1]{\ensuremath{\left\{{#1}\right\}}}
\newcommand{\indicator}[1]{\ensuremath{\mathds{1}\left\{{#1}\right\}}}
\newcommand{\condexpec}[3][]{\ensuremath{\mathbb{E}_{#1}\left[{#2} \; \middle| \; {#3} \right]}}
\newcommand{\expec}[2][]{\ensuremath{\mathbb{E}_{{#1}}\left[ {#2} \right]}}
\geometry{left=1in,right=1in,top=0.6in,bottom=1in}
\newenvironment{psmallmatrix}
{\left(\begin{smallmatrix}}
	{\end{smallmatrix}\right)}

\theoremstyle{definition}
\newtheorem{example}{Examples}[section]

\bibliographystyle{ecta}
\begin{document}
\title{Derivations, extensions, and applications for \texttt{SimpleDifferentialOperators.jl}}
\author{Presented by Chiyoung Ahn (@chiyahn)}
\maketitle

\section{Setup}

\begin{itemize}
	\item Define an irregular grid $\set{x_i}_{i=0}^{M+1}$ with \textbf{boundary nodes}, $x_0 = {x_{\min}}$ and $x_{M+1} = {x_{\max}}$. Denote the \textbf{extended grid} as $\overline{x} \equiv \set{x_i}_{i=0}^{M+1}$ and the \textbf{interior grid}, a collection of nodes excluding the boundary nodes, as $x \equiv \set{x_i}_{i=1}^{M} $.
	\item Denote the backward and forward distance between the grid points as
	\begin{align}
	\Delta_{i,-} &\equiv x_i - x_{i-1},\, \text{for } i = 1,\ldots, M+1\\
	\Delta_{i,+} &\equiv x_{i+1} - x_i,\, \text{for } i = 0,\ldots, M
	\end{align}
	\item Define the vector of backwards and forwards first differences, padding with $\Delta_{0,-} = \Delta_{M+1,+} = 0$, as
	\begin{align}
	\Delta_{-} &\equiv \begin{bmatrix} 0 \\
	\text{diff}(z)
	\end{bmatrix}\in\R^{M+2}\label{eq:Delta-minus}\\
	\Delta_{+} &\equiv \begin{bmatrix} \text{diff}(z)\\
	0
	\end{bmatrix}\in\R^{M+2}
	\end{align}
	\item Reflecting barrier conditions:
	\begin{align}
	\underline{\xi} v({x_{\min}}) + \D[x]v({x_{\min}} ) &= 0\label{eq:new-BC1}\\
	\overline{\xi} v({x_{\max}}) + \D[x]v({x_{\max}}) &= 0\label{eq:new-BC2}
	\end{align}
\end{itemize}

Let $L_{1-}$, $L_{1+}$ be the discretized backward and forward first order differential operators and $L_2$ be the discretized central difference  second order differential operator, all subject to the Neumann boundary conditions in \cref{eq:new-BC1,eq:new-BC2}, such that $L_{1-} v(x), L_{1+} v(x)$ and $L_2 v(x)$ represent the first-order (backward and forward) and second-order derivatives of $v(x)$ respectively at $x$. For second derivatives, we use the following numerical scheme from \cite{achdou17}:

\begin{equation}
v''(x_i) \approx \dfrac{ \Delta_{i,-} v( x_i + \Delta_{i,+}) - (\Delta_{i,+} + \Delta_{i,-}) v( x_i ) + \Delta_{i,+} v( x_i - \Delta_{i,-})}{\frac{1}{2}(\Delta_{i,+} + \Delta_{i,-}) \Delta_{i,+} \Delta_{i,-} }, \text{for } i = 1, \ldots, M
\end{equation}





\subsection{Regular grids}
Suppose that the grids are regular, i.e., elements of $\text{diff}(x)$ are all identical with $\Delta$ for some $\Delta > 0$.

Using the backwards first-order difference, \eqref{eq:new-BC1} implies
\begin{align}
\dfrac{v({x_{1}}) - v(x_{0})}{\Delta} &= - \underline{\xi} v({x_{0}})
\end{align}
i.e.,
\begin{align}\label{eq:BC1-extrapolation-uniform}
v(x_0) = \frac{1}{1-\underline{\xi} \Delta } v(x_1)
\end{align}
at the lower bound.

Likewise, \eqref{eq:new-BC2} under the forwards first-order difference yields
\begin{align}
\dfrac{v(x_{M+1}) - v({x_{M}})}{\Delta} &= - \overline{\xi} v({x_{M+1} })
\end{align}
i.e.,
\begin{align}\label{eq:BC2-extrapolation-uniform}
v(x_{M+1}) = \frac{1}{1+\overline{\xi} \Delta } v(x_M)
\end{align}
at the upper bound.

The discretized central difference of second order under \eqref{eq:new-BC1} at the lower bound is, by substituting \eqref{eq:BC1-extrapolation-uniform} in,
\begin{align}
\dfrac{v ({x_{1}} + \Delta) - 2 v({x_{1}}) + v(x_{\min})}{\Delta^2} &=   \dfrac{v({x_{1}} + \Delta) - v({x_{1}})}{\Delta^2} - \dfrac{1}{\Delta}\dfrac{v ({x_{1}}) - v(x_{\min}) }{\Delta}  \\
&= \dfrac{v({x_{1}} + \Delta) - v({x_{1}})}{\Delta^2} + \dfrac{1}{\Delta} \underline{\xi} v({x_{1}})  \\ 
&= \dfrac{1}{\Delta^2}  (- 1 + \Delta \underline{\xi}) v({x_{1}})  + \dfrac{1}{\Delta^2}  v({x_{1}} + \Delta)  
\end{align}
Similarly, by \eqref{eq:new-BC2}, we have
\begin{align}
\dfrac{v(x_{\max}) - 2 v({x_{M}} ) + v({x_{M}} -\Delta)}{\Delta^2} &=   \dfrac{v({x_{M}} - \Delta) - v({x_{M}})}{\Delta^2} + \dfrac{1}{\Delta}\dfrac{ v(x_{\max}) - v ({x_{M}}) }{\Delta}  \\
&= \dfrac{v({x_{M}} - \Delta) - v({x_{M}})}{\Delta^2}  - \dfrac{1}{\Delta} \overline{\xi} v({x_{M}})  \\ 
&= \dfrac{1}{\Delta^2}  (- 1 - \Delta \overline{\xi}) v({x_{M}})  + \dfrac{1}{\Delta^2}  v({x_{M}} - \Delta)  
\end{align}
at the upper bound.

Thus, the corresponding discretized differential operator $L_{1-}$, $L_{1+}$, and $L_2$ are defined as 

\begin{align}
L_{1-} &\equiv \frac{1}{\Delta}\begin{pmatrix}
1 - (1 + \underline{\xi} \Delta) &0&0&\dots&0&0&0\\
-1&1&0&\dots&0&0&0\\
\vdots&\vdots&\vdots&\ddots&\vdots&\vdots&\vdots\\
0&0&0&\dots&-1&1&0\\
0&0&0&\cdots&0&-1&1
\end{pmatrix}_{M\times M}\label{eq:L-1-regular} \\
L_{1+} &\equiv \frac{1}{\Delta}\begin{pmatrix}
-1&1&0&\dots&0&0&0\\
0&-1&1&\dots&0&0&0\\
\vdots&\vdots&\vdots&\ddots&\vdots&\vdots&\vdots\\
0&0&0&\dots&0&-1&1\\
0&0&0&\cdots&0&0&-1+(1-\overline{\xi} \Delta)
\end{pmatrix}_{M\times M}\label{eq:L-1-plus-regular} \\
L_2 &\equiv \frac{1}{\Delta^2}\begin{pmatrix}
-2 + (1 + \underline{\xi}\Delta) &1&0&\dots&0&0&0\\
1&-2&1&\dots&0&0&0\\
\vdots&\vdots&\vdots&\ddots&\vdots&\vdots&\vdots\\
0&0&0&\dots&1&-2&1\\
0&0&0&\cdots&0&1&-2 + (1- \overline{\xi}\Delta)
\end{pmatrix}_{M\times M}\label{eq:L-2-regular}
\end{align}

\subsection{Irregular grids}
Using the backwards first-order difference, \eqref{eq:new-BC1} implies
\begin{align}
\dfrac{v({x_{1}}) - v(x_{0})}{\Delta_{0,+}} &= - \underline{\xi} v({x_{0}})
\end{align}
i.e.,
\begin{align}\label{eq:BC1-extrapolation}
v(x_0) = \frac{1}{1-\underline{\xi} \Delta_{0,+} } v(x_1)
\end{align}
at the lower bound. Likewise, the forwards first-order difference under \eqref{eq:new-BC2} yields
\begin{align}
\dfrac{v(x_{M+1}) - v( {x_{M}})}{\Delta_{M+1,-}} &= - \overline{\xi} v({x_{M+1}})
\end{align}
i.e.,
\begin{align}\label{eq:BC2-extrapolation}
v(x_{M+1}) = \frac{1}{1+\overline{\xi} \Delta_{M+1,-} } v(x_M)
\end{align}
at the upper bound.

The discretized central difference of second order scheme at the lower bound under \eqref{eq:new-BC1}, by substituting \eqref{eq:BC2-extrapolation}, is 
\begin{align}
&\dfrac{\Delta_{1,-} v( {x_{1}} + \Delta_{1,+}) - (\Delta_{1,+} + \Delta_{1,-}) v({x_{1}}) + \Delta_{1,+}  v( x_{\min})}{\frac{1}{2}(\Delta_{1,+} + \Delta_{1,-}) \Delta_{1,+} \Delta_{1,-} } \\
&=
\dfrac{v ({x_{1}}+\Delta_{1, +}) - 2 v({x_{1}}) + v(x_{\min})}{\Delta_{1, +}^2} \\ 
&= \dfrac{v({x_{1}} + \Delta_{1, +}) - v({x_{1}})}{\Delta_{1, +}^2} - \dfrac{1}{\Delta_{1, +}}\dfrac{v({x_{1}}) - v(x_{\min}) }{\Delta_{1, +}}  \\
&= \dfrac{v({x_{1}} + \Delta_{1, +}) - v({x_{1}})}{\Delta_{1, +}^2} + \dfrac{1}{\Delta_{i,+}} \underline{\xi} v({x_{1}})  \\ 
&= \dfrac{1}{\Delta_{1, +}^2}  (- 1 + \Delta_{1, +} \underline{\xi}) v({x_{1}})  + \dfrac{1}{\Delta_{1, +}^2}  v({x_{1}} + \Delta_{1, +})  
\end{align}
Similarly, by \eqref{eq:new-BC2}, we have
\begin{align}
&\dfrac{\Delta_{M,-} v( x_{\max}) - (\Delta_{M,+} + \Delta_{M,-}) v({x_{M}} ) + \Delta_{M,+}  v( {x_{M}} - \Delta_{M,-})}{\frac{1}{2}(\Delta_{M,+} + \Delta_{M,-}) \Delta_{M,+} \Delta_{M,-} } \\
&=\dfrac{v(x_{\max}) - 2 v({x_{M}} ) + v({x_{M}} -\Delta_{M,-})}{\Delta_{M,-}^2} \\
&=   \dfrac{v({x_{M}} - \Delta_{M,-}) - v({x_{M}})}{\Delta_{M,-}^2} + \dfrac{1}{\Delta_{M,-}}\dfrac{ v(x_{\max}) - v ({x_{M}}) }{\Delta_{M,-}}  \\
&= \dfrac{v({x_{M}} - \Delta_{M,-}) - v({x_{M}})}{\Delta_{M,-}^2}  - \dfrac{1}{\Delta_{M,-}} \overline{\xi} v({x_{M}})  \\ 
&= \dfrac{1}{\Delta_{M,-}^2}  (- 1 - \Delta_{M,-} \overline{\xi}) v({x_{M}})  + \dfrac{1}{\Delta_{M,-}^2}  v({x_{M}} - \Delta_{M,-})  
\end{align}
at the upper bound.

Thus, the corresponding discretized differential operator $L_{1-}$, $L_{1+}$, and $L_2$ are defined as 

\begin{align}
L_{1-} &\equiv \begin{pmatrix}
\Delta^{-1}_{1,-} [1 - (1 + \underline{\xi} \Delta_{1,-})] &0&0&\dots&0&0&0\\
-\Delta_{2,-}^{-1}&\Delta_{2,-}^{-1}&0&\dots&0&0&0\\
\vdots&\vdots&\vdots&\ddots&\vdots&\vdots&\vdots\\
0&0&0&\dots&-\Delta_{M-1,-}^{-1}&\Delta_{M-1,-}^{-1}&0\\
0&0&0&\cdots&0&-\Delta_{M,-}^{-1}&\Delta_{M,-}^{-1}
\end{pmatrix}_{M\times M}\label{eq:L-1} \\
L_{1+} &\equiv \begin{pmatrix}
-\Delta_{1,+}^{-1}&\Delta_{1,+}^{-1}&0&\dots&0&0&0\\
0&-\Delta_{2,+}^{-1}&\Delta_{2,+}^{-1}&\dots&0&0&0\\
\vdots&\vdots&\vdots&\ddots&\vdots&\vdots&\vdots\\
0&0&0&\cdots&0&-\Delta_{M-1,+}^{-1}&\Delta_{M-1,+}^{-1}\\
0&0&0&\dots&0&0&\Delta_{M,+}^{-1}  [-1 + (1 - \overline{\xi} \Delta_{M,+})]  \\
\end{pmatrix}_{M\times M}\label{eq:L-1-plus} \\
L_2 &\equiv \begin{psmallmatrix}
\Delta_{1,+}^{-2}[-2 + (1+\underline{\xi} \Delta_{1,+})] &\Delta_{1,+}^{-2}&0&\cdots&0&0&0 \\
\vdots&\ddots&\ddots&\ddots&\ddots&\vdots&\vdots\\
0&\cdots&2(\Delta_{i,+}+\Delta_{i,-})^{-1} \Delta_{i,-}^{-1} &-2\Delta_{i,-}^{-1} \Delta_{i,+}^{-1}  & 2 (\Delta_{i,+}+\Delta_{i,-})^{-1} \Delta_{i,+}^{-1}&\cdots&0 \\
\vdots&\vdots&\vdots&\ddots&\ddots&\ddots&\vdots\\
0&0&0&\cdots&0&\Delta_{M,-}^{-2}&\Delta_{M,-}^{-2} [-2 + (1- \overline{\xi}\Delta_{M,-})]
\end{psmallmatrix}_{M\times M}\label{eq:L-2}
\end{align}
\iffalse
\subsection{Differential operators by basis}
Define the following basis matrices:

\begin{align}
U_1^{-} &\equiv \begin{pmatrix}
1  &0&0&\dots&0&0&0\\
-1&1&0&\dots&0&0&0\\
\vdots&\vdots&\vdots&\ddots&\vdots&\vdots&\vdots\\
0&0&0&\dots&-1&1&0\\
0&0&0&\cdots&0&-1&1
\end{pmatrix}_{M\times M}\label{eq:L-1-basis} \\
U_1^{+} &\equiv \begin{pmatrix}
-1  &1&0&\dots&0&0&0\\
0&-1&1&\dots&0&0&0\\
\vdots&\vdots&\vdots&\ddots&\vdots&\vdots&\vdots\\
0&0&0&\dots&0&-1&1\\
0&0&0&\cdots&0&0&-1
\end{pmatrix}_{M\times M}\label{eq:L-1+-basis} \\
\end{align}

and the boundary conditions for the reflecting conditions:

\begin{align}
B_{1}  &\equiv \begin{pmatrix}
(1 + \underline{\xi} \Delta^{-1}_{1,-}) &0&\dots&0&0\\
0&0&\dots&0&0\\
\vdots&\vdots&\ddots&\vdots&\vdots\\
0&0&\cdots&0&0\\
0&0&\cdots&0&0
\end{pmatrix}_{M\times M} \\
B_{M}  &\equiv \begin{pmatrix}
0 &0&\dots&0&0\\
0&0&\dots&0&0\\
\vdots&\vdots&\ddots&\vdots&\vdots\\
0&0&\cdots&0&0\\
0&0&\cdots&0&(1 - \overline{\xi} \Delta^{-1}_{M,+})
\end{pmatrix}_{M\times M}
\end{align}

\subsubsection{Regular grids}
For regular grids with the uniform distance of $\Delta > 0$, \eqref{eq:L-1-regular} and \eqref{eq:L-2-regular} can be represented by

\begin{align}
L_{1-} &= \dfrac{1}{\Delta} U_1^{-} - B_1 \\
L_{1+} &= \dfrac{1}{\Delta} U_1^{+} + B_{M} \\
L_2 &= \dfrac{1}{\Delta^2} (U_1^+ - U_1^-) + B_1 + B_{M}
\end{align}

\subsubsection{Irregular grids}
For notational brevity, for vectors with the same size, $x_1, x_2$, define $x_1 x_2$ as the elementwise-multiplied vector. Also, let $\Delta_-^\circ, \Delta_+^\circ$ as the vectors of differences on the interior nodes, i.e., $\Delta_{-}^\circ = \set{\Delta_{i,-}}_{i=1}^M$, $\Delta_{+}^\circ = \set{\Delta_{i,}}_{i=1}^M$. Then, we have
\begin{align}
L_{1-} &= \text{diag}(\Delta_{-}^{\circ} )^{-1} U_1^{-} - B_1 \\
L_{1+} &= \text{diag}(\Delta_{+}^{\circ} )^{-1} U_1^{+} + B_{M} \\
L_2 &= \text{diag} \left[ \frac{1}{2} ( {\Delta_{+}^{\circ}} + {\Delta_{-}^{\circ}} ) {\Delta_{+}^{\circ}} \right]^{-1}  U_1^{+} - 
 \text{diag} \left[ \frac{1}{2} ( {\Delta_{+}^{\circ}} + {\Delta_{-}^{\circ}} ) {\Delta_{-}^{\circ}} \right]^{-1}  U_1^{-} 
+ B_1 + B_{M} 
\end{align}
We can simplify this expression further by introducing a new notation. Let $x^{-1}$ be defined as the elementwise inverse of a vector $x$ that contains no zero element. Then, $L_2$ can be represented as
\begin{align}
L_2 &=
2\left[ \text{diag} \left( ( {\Delta_{+}^{\circ}} + {\Delta_{-}^{\circ}} )^{-1} {\Delta_{+}^{\circ}}^{-1} \right) U_1^{+} - 
\text{diag} \left( ( {\Delta_{+}^{\circ}} + {\Delta_{-}^{\circ}} )^{-1} {\Delta_{-}^{\circ}}^{-1} \right) U_1^{-}  \right]
+ B_1 + B_{M} \\ \label{eq:L-2-by-basis}
&= 2 \text{diag} \left( ( {\Delta_{+}^{\circ}} + {\Delta_{-}^{\circ}} )^{-1} \right) \left[ \text{diag} \left(  {\Delta_{+}^{\circ}}^{-1} \right) U_1^{+} - 
\text{diag} \left(  {\Delta_{-}^{\circ}}^{-1} \right) U_1^{-}  \right]
+ B_1 + B_{M}
\end{align}


The diagonal elements of \eqref{eq:L-2-by-basis} are also identical with the one provided in \eqref{eq:L-2} -- to see this, note that the diagonal elements of \eqref{eq:L-2-by-basis}, modulo $B_1$ and $B_{M}$, are
\begin{align}
-2 \left[ ({\Delta_{+}^{\circ}} + {\Delta_{-}^{\circ}})^{-1} {\Delta_{+}^{\circ}}^{-1} + ({\Delta_{+}^{\circ}} + {\Delta_{-}^{\circ}})^{-1} {\Delta_{-}^{\circ}}^{-1} \right] &= -2 ({\Delta_{+}^{\circ}} + {\Delta_{-}^{\circ}})^{-1}  ( {\Delta_{+}^{\circ}}^{-1} + {\Delta_{-}^{\circ}}^{-1} ) \\
&= -2({\Delta_{+}^{\circ}} + {\Delta_{-}^{\circ}})^{-1} ({\Delta_{+}^{\circ}}^{-1} {\Delta_{-}^{\circ}}^{-1}) ({\Delta_{+}^{\circ}} + {\Delta_{-}^{\circ}} )  \\
&= -2 ({\Delta_{+}^{\circ}}^{-1} {\Delta_{-}^{\circ}}^{-1})
\end{align}
which is identical with $\text{diag} (L_2)$ with $L_2$ from \eqref{eq:L-2} except the first row and last row that are affected by $B_1$ and $B_{M}$.

\fi 

\section{Boundary conditions with operators on extended grids}
Boundary conditions can be applied manually by using operators on extended grids, $\overline{x}$, to find solutions on extended grids. Define $v(\overline{x})$ as $(M+2)$-vector whose $i$th element is $v(\overline {x}_i)$. We can then define the following operators on $\overline{x}$:


\begin{equation}\label{eq:L-1-minus-extended}
\overline{L}_{1-} \equiv\begin{pmatrix}
-\Delta_{1,-}^{-1}&\Delta_{1,-}^{-1}&0&\dots&0&0&0\\
0&-\Delta_{2,-}^{-1}&\Delta_{2,-}^{-1}&\dots&0&0&0\\
\vdots&\vdots&\vdots&\ddots&\vdots&\vdots&\vdots\\
0&0&0&\dots&\Delta_{M-1,-}^{-1}&0&0\\
0&0&0&\cdots&-\Delta_{M,-}^{-1}&\Delta_{M,-}^{-1}&0
\end{pmatrix}_{M\times (M+2)}
\end{equation}

\begin{equation}\label{eq:L-1-plus-extended}
\overline{L}_{1+} \equiv \begin{pmatrix}
0&-\Delta_{1,+}^{-1}&\Delta_{1,+}^{-1}&\dots&0&0&0\\
0&0&-\Delta_{2,+}^{-1}&\dots&0&0&0\\
\vdots&\vdots&\vdots&\ddots&\vdots&\vdots&\vdots\\
0&0&0&\dots&-\Delta_{M-1,+}^{-1}&\Delta_{M-1,+}^{-1}&0\\
0&0&0&\cdots&0&-\Delta_{M,+}^{-1}&\Delta_{M,+}^{-1}&
\end{pmatrix}_{M\times (M+2)}
\end{equation}


\begin{equation}\label{eq:L-2-extended} \small
\overline{L}_2 \equiv \frac{1}{\Delta^2}\begin{psmallmatrix}
\Delta_{1,-}^{-2} &-2\Delta_{1,-}^{-1} \Delta_{1,+}^{-1}  & \Delta_{i,+}^{-2}&\dots&0&0&0\\
0&2(\Delta_{2,+}+\Delta_{2,-})^{-1} \Delta_{2,-}^{-1} &-2\Delta_{2,-}^{-1} \Delta_{2,+}^{-1} &\dots&0&0&0\\
\vdots&\vdots&\vdots&\ddots&\vdots&\vdots&\vdots\\
0&0&0&\dots&-2\Delta_{M-1,-}^{-1} \Delta_{M-1,+}^{-1}  & 2 (\Delta_{M-1,+}+\Delta_{M-1,-})^{-1} \Delta_{M-1,+}^{-1}&0\\
0&0&0&\cdots&\Delta_{M,-}^{-2} &-2\Delta_{M,-}^{-1} \Delta_{M,+}^{-1}  & \Delta_{M,+}^{-2}
\end{psmallmatrix}_{M\times (M+2)}
\end{equation}


Suppose that we want to solve a system $L v({x}) = f(x) $ for $v(x)$ where $L$ is a linear combination of discretized differential operators for some $f(x)$ that represents the values of a function $f$ on discretized $x$. To solve the system under boundary conditions on $v$, one can construct and solve the following extended system:

\begin{equation}\label{eq:reflecting-barrier-extended-system}
\begin{bmatrix}
\overline{L} \\
B
\end{bmatrix} 
v(\overline{x}) = 
\begin{bmatrix}
f(x) \\
b
\end{bmatrix} 
\end{equation}

with $M_E$ by $(M+2)$ matrix $B$ and $M_E$-length vector $b$ that represent the current boundary conditions, where $M_E$ is the number of boundary conditions to be applied. The solution of \eqref{eq:reflecting-barrier-extended-system}, $v(\overline{x})$ can be decomposed into 
\begin{align}
v(\overline{x}) = \begin{bmatrix}
v(x_{\min}) \\
v(x) \\
v(x_{\max})
\end{bmatrix}
\end{align}
which also gives the solution for $v(x)$.

\subsection{Reflecting barrier conditions}

To apply reflecting barrier conditions $v'(x_{\min}) = v'(x_{\max}) = 0$, one can use

\begin{equation}\label{eq:reflecting-barrier-matrix}
B = \begin{bmatrix}
-1 & 1 & 0 & \dots & 0 & 0 & 0 \\
0 & 0 & 0 & \dots & 0 & -1 & 1\\
\end{bmatrix}_{2 \times (M+2)} \quad 
b = \begin{bmatrix}
0 \\
0
\end{bmatrix}
\end{equation}

\subsubsection{Gaussian elimination}
Instead of solving \eqref{eq:reflecting-barrier-extended-system} for a value function $v(\overline{x})$ on the extended grid, one can perform Gaussian elimination to reduce the system and solve $v(x)$, which gives the identical solution as the interior of $v(\overline{x})$. 

\begin{example}\label{ex:gaussian-elimination-reflecting-barrier}
	 Consider $L = L_{2}$ to solve $L v = f(x)$ with $M = 3$ under uniform grids $\overline{x} = \{x_0, x_1, x_2, x_3, x_4\}$ and $\Delta = 1$, whose corresponding interior grid is $x = \{x_1, x_2, x_3\}$. This gives
	 \begin{align}
	 L = 	 \begin{bmatrix}
	 -2 & 1 & 0 \\
	 1 & -2 & 1 \\
	 0 & 1 & -2 \\
	 \end{bmatrix}
	 \end{align}
	 so $L v= f(x)$ on the grid $x$ results in the following system
	 \begin{align} \label{eq:extended-system-reflecting-barrier-reduced-system}
	 	 	 \begin{bmatrix}
	 -1 & 1 & 0  \\
	 1 & -2 & 1 \\
	 0 & 1 & -1 \\
	 \end{bmatrix} 	  \begin{bmatrix}
	 v(x_1) \\
	 v(x_2) \\
	 v(x_3) 
	 \end{bmatrix}
	 = 
	 \begin{bmatrix}
	 f(x_1) \\
	 f(x_2) \\
	 f(x_3) 
	 \end{bmatrix} 
	 \end{align}
	 
	 
	 For the extended system we have
	 \begin{align}
	 \overline{L} &= 
	 \begin{bmatrix}
	 1 & -2 & 1 & 0 & 0 \\
	 0 & 1 & -2 & 1 & 0 \\
	 0 & 0 & 1 & -2 & 1 \\
	 \end{bmatrix} \\
	 B &= \begin{bmatrix}
  	-1 & 1  & 0 & 0 & 0 \\
	0 & 0 & 0 & -1 & 1
	 \end{bmatrix}
	 \end{align}
	 
	  Constructing the stacked extended system \eqref{eq:reflecting-barrier-extended-system} gives
	  \begin{align}\label{eq:extended-system-reflecting-barrier-before-gaussian-elimination}
 	 \begin{bmatrix}
	  1 & -2 & 1 & 0 & 0 \\
      0 & 1 & -2 & 1 & 0 \\
	  0 & 0 & 1 & -2 & 1 \\
	  -1 & 1  & 0 & 0 & 0 \\
	  0 & 0 & 0 & -1 & 1
	  \end{bmatrix} 
	  \begin{bmatrix}
	  v(x_0) \\
	  v(x_1) \\
	  v(x_2) \\
	  v(x_3) \\
	  v(x_4) 
	  \end{bmatrix}
	  = 
 		\begin{bmatrix}
	  f(x_1) \\
	  f(x_2) \\
	  f(x_3) \\
	  0 \\
	  0 
	  \end{bmatrix} 
	  \end{align}
	  
	 Note that substracting the first row of $\overline{L}$ by $(-1)$ times the first row of $B$ returns an identical system as \eqref{eq:extended-system-reflecting-barrier-before-gaussian-elimination}. Likewise, substracting the last row of $\overline{L}$ by $(-1)$ times the last row of $B$ returns the identical system. Performing the two Gaussian elimination yields the following system:
	 \begin{align}\label{eq:extended-system-reflecting-barrier-after-gaussian-elimination}
	 \begin{bmatrix}
	 0 & -1 & 1 & 0 & 0 \\
     0 & 1 & -2 & 1 & 0 \\
     0 & 0 & 1 & -1 & 0 \\
	 1 & -1  & 0 & 0 & 0 \\
	 0 & 0 & 0 & -1 & 1
	 \end{bmatrix} 	  \begin{bmatrix}
	 v(x_0) \\
	 v(x_1) \\
	 v(x_2) \\
	 v(x_3) \\
	 v(x_4)  \\
	 \end{bmatrix}
	 = 
	 \begin{bmatrix}
	 f(x_2) \\
	 f(x_3) \\
	 f(x_4) \\
	 0 \\
	 0 
	 \end{bmatrix} 
	 \end{align}
	 Note that now we have the first three rows of the coefficient matrix with zero columns on the extended nodes, $v(x_1 - \Delta)$ and $v(x_3 + \Delta)$. Extracting the system corresponding to the first three rows returns the following system, which solves the interior of $v(\overline{x})$, i.e., $v(x)$:
	 \begin{align}
	 	 \begin{bmatrix}
	 -1 & 1 & 0  \\
	 1 & -2 & 1 \\
	 0 & 1 & -1 \\
	 \end{bmatrix} 	  \begin{bmatrix}
	 v(x_1) \\
	 v(x_2) \\
	 v(x_3) 
	 \end{bmatrix}
	 = 
	 \begin{bmatrix}
	 f(x_1) \\
	 f(x_2) \\
	 f(x_3) 
	 \end{bmatrix} 
	 \end{align}
	 which is identical as  \eqref{eq:extended-system-reflecting-barrier-reduced-system}.
\end{example}


\subsection{Mixed barrier conditions}

To apply a mixed barrier condition $v'(x_{\min}) = - \underline{\xi} v(x_{\min}) $ for some $\xi \in \mathbb{R}$, with one reflecting barrier condition on the upper bound $v'(x_{\max}) = 0$ and $\Delta_{1,-} = x_1 - x_0$ one can use

\begin{equation}\label{eq:mixed-barrier-matrix-original}
B = \begin{bmatrix}
\dfrac{1}{\Delta_{0,+}} - \underline{\xi} & \dfrac{1}{\Delta_{0,+}} & 0 & \dots & 0 & 0 & 0 \\
0 & 0 & 0 & \dots & 0 & -1 & 1\\
\end{bmatrix}_{2 \times (M+2)} \quad 
b = \begin{bmatrix}
0 \\
0
\end{bmatrix}
\end{equation}
from the discretized scheme
\begin{align}
\dfrac{v(x_1) - v(x_0)}{\Delta_{0,+}} = - \underline{\xi} v(x_0)
\end{align}
for $v'(x_{\min}) = - \underline{\xi} v(x_{\min})$. Or equivalently,
\begin{equation}\label{eq:mixed-barrier-matrix}
B = \begin{bmatrix}
1 - \underline{\xi} \Delta_{1,-} & \Delta_{1,-} & 0 & \dots & 0 & 0 & 0 \\
0 & 0 & 0 & \dots & 0 & -1 & 1\\
\end{bmatrix}_{2 \times (M+2)} \quad 
b = \begin{bmatrix}
0 \\
0
\end{bmatrix}
\end{equation}

Similarly, one can apply a mixed barrier condition $v'(x_{\min}) = - \underline{\xi} v(x_{\min}) $ for some $\xi \in \mathbb{R}$ on the upper bound, with one reflecting barrier condition on the lower bound $v'(x_{\min}) = 0$ and $\Delta_{M+1,-} = x_{M+1} - x_M$ with


\begin{equation}\label{eq:mixed-barrier-matrix-ub}
B = \begin{bmatrix}
-1 & 1 & 0 & \dots & 0 & 0 & 0 \\
0 & 0 & 0 & \dots & 0 & -\Delta_{M+1,-} & 1 + \overline{\xi} \Delta_{M+1,-}   \\
\end{bmatrix}_{2 \times (M+2)} \quad 
b = \begin{bmatrix}
0 \\
S
\end{bmatrix}
\end{equation}



\subsection{Absorbing barrier conditions}

To apply an absorbing barrier condition $v(x_{\min}) =S$ for some $S \in \mathbb{R}$, with one reflecting barrier condition on the upper bound $v'(x_{\max}) = 0$, one can use

\begin{equation}\label{eq:absorbing-barrier-matrix}
B = \begin{bmatrix}
1 & 0 & 0 & \dots & 0 & 0 & 0 \\
0 & 0 & 0 & \dots & 0 & -1 & 1\\
\end{bmatrix}_{2 \times (M+2)} \quad 
b = \begin{bmatrix}
S \\
0
\end{bmatrix}
\end{equation}

Similarly, one can apply an absorbing condition on the upper bound  $v(x_{\max}) =S$ for some $S \in \mathbb{R}$ and the 


\begin{equation}\label{eq:absorbing-barrier-matrix-ub}
B = \begin{bmatrix}
-1 & 1 & 0 & \dots & 0 & 0 & 0 \\
0 & 0 & 0 & \dots & 0 & 0 & 1\\
\end{bmatrix}_{2 \times (M+2)} \quad 
b = \begin{bmatrix}
0 \\
S
\end{bmatrix}
\end{equation}

\subsubsection{Gaussian elimination}

\begin{example}
	Consider solving \eqref{eq:extended-system-reflecting-barrier-reduced-system}, but this time with an absorbing barrier condition on the lower bound, $v(x_{\min}) = S$ with a boundary condition matrix 
	\begin{align}
	B &= \begin{bmatrix}
			1 & 0  & 0 & 0 & 0 \\
			0 & 0 & 0 & -1 & 1
		\end{bmatrix}
	\end{align}
	The corresponding extended system is
\begin{align}\label{eq:extended-system-absorbing-barrier-before-gaussian-elimination}
\begin{bmatrix}
1 & -2 & 1 & 0 & 0 \\
0 & 1 & -2 & 1 & 0 \\
0 & 0 & 1 & -2 & 1 \\
1 & 0  & 0 & 0 & 0 \\
0 & 0 & 0 & -1 & 1
\end{bmatrix} 
\begin{bmatrix}
v(x_0) \\
v(x_1) \\
v(x_2) \\
v(x_3) \\
v(x_4) \\
\end{bmatrix}
= 
\begin{bmatrix}
f(x_1) \\
f(x_2) \\
f(x_3) \\
S \\
0 
\end{bmatrix} 
\end{align}

Note that substracting the first row of $\overline{L}$ by $(-1)$ times the first row of $B$ returns an identical system as \eqref{eq:extended-system-absorbing-barrier-before-gaussian-elimination}. Likewise, substracting the last row of $\overline{L}$ by $(-1)$ times the last row of $B$ returns the identical system. Performing the two Gaussian elimination yields the following system:
\begin{align}\label{eq:extended-system-absorbing-barrier-after-gaussian-elimination}
\begin{bmatrix}
0 & -2 & 1 & 0 & 0 \\
0 & 1 & -2 & 1 & 0 \\
0 & 0 & 1 & -1 & 0 \\
1 & 0  & 0 & 0 & 0 \\
0 & 0 & 0 & -1 & 1
\end{bmatrix} 	  \begin{bmatrix}
v(x_0) \\
v(x_1) \\
v(x_2) \\
v(x_3) \\
v(x_4) \\
\end{bmatrix}
= 
\begin{bmatrix}
f(x_1) - S \\
f(x_2) \\
f(x_3) \\
0 \\
0 
\end{bmatrix} 
\end{align}

\end{example}

Extracting the system corresponding to the first three rows returns the following system, which solves the interior of $v(\overline{x})$, i.e., $v(x)$:
\begin{align}
\begin{bmatrix}
-2 & 1 & 0  \\
1 & -2 & 1 \\
0 & 1 & -1 \\
\end{bmatrix} 	  \begin{bmatrix}
v(x_1) \\
v(x_2) \\
v(x_3) 
\end{bmatrix}
= 
\begin{bmatrix}
f(x_1) - S \\
f(x_2) \\
f(x_3) 
\end{bmatrix} 
\end{align}

\section{Applications}

\subsection{Hamilton–Jacobi–Bellman equations (HJBE)}
Consider solving for $v$ from the following optimal control problem
\begin{align}
v(x_0) = \max_{ {\{\alpha(t) \} }_{t \geq 0} } \int_{0}^\infty e^{-\rho t} r( x(t), \alpha(t )) dt
\end{align}

with the law of motion for the state 
\begin{align}
dx = \mu dt + \sigma dW 
\end{align}


for some constant $\mu \geq 0$ and $\sigma \geq 0$ with $x(0) = x_0$.

Let $\alpha^*(t)$ be the optimal solution. Suppose that $r$ under $\alpha^*(t)$ can be expressed in terms of state variables, $r^* (x)$. Then, the HJBE yields

\begin{equation}\label{eq:hamilton-jacobi-bellman}
\rho v(x) = r^*(x) +  \mu  \partial_{x} v(x) + \dfrac{\sigma^2}{2} \partial_{xx} v(x)
\end{equation}

In terms of differential operators, one can rewrite the equation as
\begin{equation}\label{eq:hjbe-system-function}
\tilde{L}  v(x) = r^*(x)
\end{equation}

with $\tilde{L} = \rho -  \tilde{L}_x$ where $\tilde{L}_x$ is defined as

\begin{equation}\label{eq:L-defn}
\tilde{L}_x = \mu \partial_{x} + (\sigma^2/2) \partial_{xx}
\end{equation}


By descretizing the space of $x$, one can solve the corresponding system by using discretized operators for $\partial_{x}$ ($L_{1+}$), $\partial_{xx}$ ($L_2$) on some grids of length $M$, $\{x_i\}_{i=1}^M$:
\begin{align}\label{eq:L-descritized-defn}
L_x = \mu L_{1+} + \dfrac{\sigma^2}{2} L_{2}
\end{align}
so that $v$ from \eqref{eq:hjbe-system-function} can be computed by solving the following discretized system of equations:
\begin{align}
L v &= r^*
\end{align}
where $v$ and $r^*$ are $M$-vectors whose $i$th elements are $v(x_i)$ and $r^*(x_i)$, respectively, and $L$ is defined as $L = \rho I - L_x$.



\subsection{Kolmogorov forward equations (KFE) under diffusion process}
Let $g(x, t)$ be the distribution of $x$ at time $t$ from the example above. By the Kolmogorov forward equation, the following PDE holds:

\begin{equation}\label{eq:kfe}
\partial_{t} g(x, t) = - \mu \partial_{x}  g(x,t) + \dfrac{\sigma^2}{2} \partial_{xx} g(x,t)
\end{equation}

\subsubsection{Stationary distributions}
The stationary distribution $g^*(x)$ satisfies

\begin{equation}\label{eq:kfe-stationary}
0 = - \mu \partial_{x} g^*(x) + \dfrac{\sigma^2}{2} \partial_{xx} g^*(x)
\end{equation}

which can be rewritten as 

\begin{equation}
\tilde{L}^*_x g(x) = 0
\end{equation}

where 

\begin{equation}\label{eq:L_star-defn}
\tilde{L}^*_x =  - \mu \partial_{x} + (\sigma^2/2) \partial_{xx}
\end{equation}

By descretizing the space of $x$, one can solve the corresponding system by using discretized operators for $\tilde{L}^*_x$. Note that the operator for the KFE in \eqref{eq:L_star-defn} is the adjoint operator of the operator for the HJBE in \eqref{eq:L-defn}, and the correct discretization scheme for \eqref{eq:L_star-defn} can be, analogously, done by taking the transpose of the discretized operator for HJBE in \eqref{eq:L-descritized-defn} -- see \cite{gabaix16} and \cite{achdou17}. Hence, one can find the stationary distribution by solving the following discretized system of equations:

\begin{equation}
L^T_x g = 0 
\end{equation}
where $L^T_x$ is the transpose of $L_x$ from \eqref{eq:L-descritized-defn} and $g$ is an $M$-vector whose element is $g(x_i)$ such that $\sum_{i=1}^M g(x_i) = 1$.

\subsubsection{Full dynamics of distributions}
One can also solve the full PDE in \eqref{eq:kfe}, given an initial distribution $g(x, 0)$. After discretization, note that \eqref{eq:kfe} can be rewritten as
\begin{equation}\label{eq:kfe-discretized}
\dot{g}(t) = L^T_x g(t)
\end{equation}
where $\dot{g}(t)$ is an $M$-vector whose $i$th element is $\partial_{t} g(x_i, t)$, which can be efficently solved by a number of differential equation solvers available in public, including \cite{rackauckas17}.

\newpage
\bibliography{hact}
% The notebook is based on my previous version that can be found at https://github.com/ubcecon/computing_and_datascience/blob/a33d14191afba2d13598b331b3623a9512dccc90/continuous_time_methods/notes/discretized-differential-operator-derivation.tex by @chiyahn
\end{document}
