% !TEX program = pdflatex

\documentclass[11pt]{article}
\usepackage{amsmath,amsfonts,amsthm,amssymb,geometry,dsfont}
\usepackage[usenames,dvipsnames,svgnamesable]{xcolor}
\usepackage[capitalise,noabbrev]{cleveref} %
\usepackage{natbib}
\crefname{equation}{}{} %
\crefname{assumption}{Assumption}{Assumptions}
\crefname{property}{Property}{Properties}
\geometry{left=1in,right=1in,top=0.6in,bottom=1in}

\newcommand{\D}[1][]{\ensuremath{\boldsymbol{\partial}_{#1}}}
\newcommand{\R}{\ensuremath{\mathbb{R}}}
\newcommand{\diff}{\ensuremath{\mathrm{d}}}
\newcommand{\set}[1]{\ensuremath{\left\{{#1}\right\}}}
\newcommand{\indicator}[1]{\ensuremath{\mathds{1}\left\{{#1}\right\}}}
\newcommand{\condexpec}[3][]{\ensuremath{\mathbb{E}_{#1}\left[{#2} \; \middle| \; {#3} \right]}}
\newcommand{\expec}[2][]{\ensuremath{\mathbb{E}_{{#1}}\left[ {#2} \right]}}
\geometry{left=1in,right=1in,top=0.6in,bottom=1in}
\newenvironment{psmallmatrix}
{\left(\begin{smallmatrix}}
	{\end{smallmatrix}\right)}
\bibliographystyle{ecta}
\begin{document}
\title{Derivations, extensions, and applications for \texttt{SimpleDifferentialOperators.jl}}
\author{Presented by Chiyoung Ahn (@chiyahn)}
\maketitle

\section{Setup}

\begin{itemize}
	\item Define an irregular grid $\set{x_i}_{i=1}^M$ with $x_1 = {x_{\min}}$ and $x_{M} = {x_{\max}}$. Denote the grid with the variable name, i.e. $x \equiv \set{x_i}_{i=1}^M$.
	\item Denote the distance between the grid points as
	\begin{align}
	\Delta_{i,-} &\equiv x_i - x_{i-1},\, \text{for } i = 2,\ldots, M\\
	\Delta_{i,+} &\equiv x_{i+1} - x_i,\, \text{for } i = 1,\ldots, M-1
	\end{align}
	
	\item Assume $\Delta_{1, -} = \Delta_{1, +}$ and $\Delta_{M, +} = \Delta_{M, -}$, due to ghost points, $x_0$ and $x_{M+1}$ on both boundaries. (i.e. the distance to the ghost nodes are the same as the distance to the closest nodes).  Then define the vector of backwards and forwards first differences as
	\begin{align}
	\Delta_{-} &\equiv \begin{bmatrix} \Delta_{1,-} \\
	\text{diff}(x)
	\end{bmatrix}\\
	\Delta_{+} &\equiv \begin{bmatrix} \text{diff}(x)\\
	\Delta_{M,+}
	\end{bmatrix}
	\end{align}
	\item Reflecting barrier conditions:
	\begin{align}
	\underline{\xi} v({x_{\min}}) + \D[x]v({x_{\min}} ) &= 0\label{eq:new-BC1}\\
	\overline{\xi} v({x_{\max}}) + \D[x]v({x_{\max}}) &= 0\label{eq:new-BC2}
	\end{align}
\end{itemize}

Let $L_{1-}$, $L_{1+}$ be the discretized backward and forward first order differential operators and $L_2$ be the discretized central difference  second order differential operator, all subject to the Neumann boundary conditions in \cref{eq:new-BC1,eq:new-BC2}, such that $L_{1-} v(x), L_{1+} v(x)$ and $L_2 v(x)$ represent the first-order (backward and forward) and second-order derivatives of $v(x)$ respectively at $x$. For second derivatives, we use the following numerical scheme from \cite{achdou17}:

\begin{equation}
v''(x_i) \approx \dfrac{ \Delta_{i,-} v( x_i + \Delta_{i,+}) - (\Delta_{i,+} + \Delta_{i,-}) v( x_i ) + \Delta_{i,+} v( x_i - \Delta_{i,-})}{\frac{1}{2}(\Delta_{i,+} + \Delta_{i,-}) \Delta_{i,+} \Delta_{i,-} }, \text{for } i = 1, \ldots, M
\end{equation}





\subsection{Regular grids}
Suppose that the grids are regular, i.e., elements of $\text{diff}(x)$ are all identical with $\Delta$ for some $\Delta > 0$.

Using the backwards first-order difference, \eqref{eq:new-BC1} implies
\begin{align}
\dfrac{v({x_{\min}}) - v({x_{\min}}-\Delta)}{\Delta} &= - \underline{\xi} v({x_{\min}})
\end{align}
at the lower bound.

Likewise, \eqref{eq:new-BC2} under the forwards first-order difference yields
\begin{align}
\dfrac{v({x_{\max}} + \Delta) - v({x_{\max}})}{\Delta} &= - \overline{\xi} v({x_{\max}})
\end{align}
at the upper bound.

The discretized central difference of second order under \eqref{eq:new-BC1} at the lower bound is
\begin{align}
\dfrac{v ({x_{\min}} + \Delta) - 2 v({x_{\min}}) + v({x_{\min}}-\Delta)}{\Delta^2} &=   \dfrac{v({x_{\min}} + \Delta) - v({x_{\min}})}{\Delta^2} - \dfrac{1}{\Delta}\dfrac{v ({x_{\min}}) - v({x_{\min}}-\Delta) }{\Delta}  \\
&= \dfrac{v({x_{\min}} + \Delta) - v({x_{\min}})}{\Delta^2} + \dfrac{1}{\Delta} \underline{\xi} v({x_{\min}})  \\ 
&= \dfrac{1}{\Delta^2}  (- 1 + \Delta \underline{\xi}) v({x_{\min}})  + \dfrac{1}{\Delta^2}  v({x_{\min}} + \Delta)  
\end{align}
Similarly, by \eqref{eq:new-BC2}, we have
\begin{align}
\dfrac{v ({x_{\max}} + \Delta) - 2 v({x_{\max}} ) + v({x_{\max}} -\Delta)}{\Delta^2} &=   \dfrac{v({x_{\max}} - \Delta) - v({x_{\max}})}{\Delta^2} + \dfrac{1}{\Delta}\dfrac{ v({x_{\max}}+\Delta) - v ({x_{\max}}) }{\Delta}  \\
&= \dfrac{v({x_{\max}} - \Delta) - v({x_{\max}})}{\Delta^2}  - \dfrac{1}{\Delta} \overline{\xi} v({x_{\max}})  \\ 
&= \dfrac{1}{\Delta^2}  (- 1 - \Delta \overline{\xi}) v({x_{\max}})  + \dfrac{1}{\Delta^2}  v({x_{\max}} - \Delta)  
\end{align}
at the upper bound.

Thus, the corresponding discretized differential operator $L_{1-}$, $L_{1+}$, and $L_2$ are defined as 

\begin{align}
L_{1-} &\equiv \frac{1}{\Delta}\begin{pmatrix}
1 - (1 + \underline{\xi} \Delta) &0&0&\dots&0&0&0\\
-1&1&0&\dots&0&0&0\\
\vdots&\vdots&\vdots&\ddots&\vdots&\vdots&\vdots\\
0&0&0&\dots&-1&1&0\\
0&0&0&\cdots&0&-1&1
\end{pmatrix}_{M\times M}\label{eq:L-1-regular} \\
L_{1+} &\equiv \frac{1}{\Delta}\begin{pmatrix}
-1&1&0&\dots&0&0&0\\
0&-1&1&\dots&0&0&0\\
\vdots&\vdots&\vdots&\ddots&\vdots&\vdots&\vdots\\
0&0&0&\dots&0&-1&1\\
0&0&0&\cdots&0&0&-1+(1-\overline{\xi} \Delta)
\end{pmatrix}_{M\times M}\label{eq:L-1-plus-regular} \\
L_2 &\equiv \frac{1}{\Delta^2}\begin{pmatrix}
-2 + (1 + \underline{\xi}\Delta) &1&0&\dots&0&0&0\\
1&-2&1&\dots&0&0&0\\
\vdots&\vdots&\vdots&\ddots&\vdots&\vdots&\vdots\\
0&0&0&\dots&1&-2&1\\
0&0&0&\cdots&0&1&-2 + (1- \overline{\xi}\Delta)
\end{pmatrix}_{M\times M}\label{eq:L-2-regular}
\end{align}

\subsection{Irregular grids}
Using the backwards first-order difference, \eqref{eq:new-BC1} implies
\begin{align}
\dfrac{v({x_{\min}}) - v({x_{\min}}-\Delta_{1, -})}{\Delta_{1, -}} &= - \underline{\xi} v({x_{\min}})
\end{align}

at the lower bound. Likewise, the forwards first-order difference under \eqref{eq:new-BC2} yields
\begin{align}
\dfrac{v({x_{\max}} + \Delta_{M, +}) - v( {x_{\max}})}{\Delta_{M, +}} &= - \overline{\xi} v({x_{\max}})
\end{align}
at the upper bound.

Note that we have assumed that $\Delta_{1,-} = \Delta_{1,+}$ and $\Delta_{M,+} = \Delta_{M,-}$ for the ghost notes. The discretized central difference of second order scheme at the lower bound under \eqref{eq:new-BC1} is 
\begin{align}
&\dfrac{\Delta_{1,-} v( {x_{\min}} + \Delta_{1,+}) - (\Delta_{1,+} + \Delta_{1,-}) v({x_{\min}}) + \Delta_{1,+}  v( {x_{\min}} - \Delta_{1,-})}{\frac{1}{2}(\Delta_{1,+} + \Delta_{1,-}) \Delta_{1,+} \Delta_{1,-} } \\
&=
\dfrac{v ({x_{\min}}+\Delta_{1, +}) - 2 v({x_{\min}}) + v({x_{\min}}-\Delta_{1, +})}{\Delta_{1, +}^2} \\ 
&= \dfrac{v({x_{\min}} + \Delta_{1, +}) - v({x_{\min}})}{\Delta_{1, +}^2} - \dfrac{1}{\Delta_{1, +}}\dfrac{v({x_{\min}}) - v({x_{\min}}-\Delta_{1, +}) }{\Delta_{1, +}}  \\
&= \dfrac{v({x_{\min}} + \Delta_{1, +}) - v({x_{\min}})}{\Delta_{1, +}^2} + \dfrac{1}{\Delta_{i,+}} \underline{\xi} v({x_{\min}})  \\ 
&= \dfrac{1}{\Delta_{1, +}^2}  (- 1 + \Delta_{1, +} \underline{\xi}) v({x_{\min}})  + \dfrac{1}{\Delta_{1, +}^2}  v({x_{\min}} + \Delta_{1, +})  
\end{align}
Similarly, by \eqref{eq:new-BC2}, we have
\begin{align}
&\dfrac{\Delta_{M,-} v( {x_{\max}} + \Delta_{M,+}) - (\Delta_{M,+} + \Delta_{M,-}) v({x_{\max}} ) + \Delta_{M,+}  v( {x_{\max}} - \Delta_{M,-})}{\frac{1}{2}(\Delta_{M,+} + \Delta_{M,-}) \Delta_{M,+} \Delta_{M,-} } \\
&=\dfrac{v ({x_{\max}} + \Delta_{M,-}) - 2 v({x_{\max}} ) + v({x_{\max}} -\Delta_{M,-})}{\Delta_{M,-}^2} \\
&=   \dfrac{v({x_{\max}} - \Delta_{M,-}) - v({x_{\max}})}{\Delta_{M,-}^2} + \dfrac{1}{\Delta_{M,-}}\dfrac{ v({x_{\max}}+\Delta_{M,-}) - v ({x_{\max}}) }{\Delta_{M,-}}  \\
&= \dfrac{v({x_{\max}} - \Delta_{M,-}) - v({x_{\max}})}{\Delta_{M,-}^2}  - \dfrac{1}{\Delta_{M,-}} \overline{\xi} v({x_{\max}})  \\ 
&= \dfrac{1}{\Delta_{M,-}^2}  (- 1 - \Delta_{M,-} \overline{\xi}) v({x_{\max}})  + \dfrac{1}{\Delta_{M,-}^2}  v({x_{\max}} - \Delta_{M,-})  
\end{align}
at the upper bound.

Thus, the corresponding discretized differential operator $L_{1-}$, $L_{1+}$, and $L_2$ are defined as 

\begin{align}
L_{1-} &\equiv \begin{pmatrix}
\Delta^{-1}_{1,-} [1 - (1 + \underline{\xi} \Delta_{1,-})] &0&0&\dots&0&0&0\\
-\Delta_{2,-}^{-1}&\Delta_{2,-}^{-1}&0&\dots&0&0&0\\
\vdots&\vdots&\vdots&\ddots&\vdots&\vdots&\vdots\\
0&0&0&\dots&-\Delta_{M-1,-}^{-1}&\Delta_{M-1,-}^{-1}&0\\
0&0&0&\cdots&0&-\Delta_{M,-}^{-1}&\Delta_{M,-}^{-1}
\end{pmatrix}_{M\times M}\label{eq:L-1} \\
L_{1+} &\equiv \begin{pmatrix}
-\Delta_{1,+}^{-1}&\Delta_{1,+}^{-1}&0&\dots&0&0&0\\
0&-\Delta_{2,+}^{-1}&\Delta_{2,+}^{-1}&\dots&0&0&0\\
\vdots&\vdots&\vdots&\ddots&\vdots&\vdots&\vdots\\
0&0&0&\cdots&0&-\Delta_{M-1,+}^{-1}&\Delta_{M-1,+}^{-1}\\
0&0&0&\dots&0&0&\Delta_{M,+}^{-1}  [-1 + (1 - \overline{\xi} \Delta_{M,+})]  \\
\end{pmatrix}_{M\times M}\label{eq:L-1-plus} \\
L_2 &\equiv \begin{psmallmatrix}
\Delta_{1,+}^{-2}[-2 + (1+\underline{\xi} \Delta_{1,+})] &\Delta_{1,+}^{-2}&0&\cdots&0&0&0 \\
\vdots&\ddots&\ddots&\ddots&\ddots&\vdots&\vdots\\
0&\cdots&2(\Delta_{i,+}+\Delta_{i,-})^{-1} \Delta_{i,-}^{-1} &-2\Delta_{i,-}^{-1} \Delta_{i,+}^{-1}  & 2 (\Delta_{i,+}+\Delta_{i,-})^{-1} \Delta_{i,+}^{-1}&\cdots&0 \\
\vdots&\vdots&\vdots&\ddots&\ddots&\ddots&\vdots\\
0&0&0&\cdots&0&\Delta_{M,-}^{-2}&\Delta_{M,-}^{-2} [-2 + (1- \overline{\xi}\Delta_{M,-})]
\end{psmallmatrix}_{M\times M}\label{eq:L-2}
\end{align}

\subsection{Differential operators by basis}
Define the following basis matrices:

\begin{align}
U_1^{-} &\equiv \begin{pmatrix}
1  &0&0&\dots&0&0&0\\
-1&1&0&\dots&0&0&0\\
\vdots&\vdots&\vdots&\ddots&\vdots&\vdots&\vdots\\
0&0&0&\dots&-1&1&0\\
0&0&0&\cdots&0&-1&1
\end{pmatrix}_{M\times M}\label{eq:L-1-basis} \\
U_1^{+} &\equiv \begin{pmatrix}
-1  &1&0&\dots&0&0&0\\
0&-1&1&\dots&0&0&0\\
\vdots&\vdots&\vdots&\ddots&\vdots&\vdots&\vdots\\
0&0&0&\dots&0&-1&1\\
0&0&0&\cdots&0&0&-1
\end{pmatrix}_{M\times M}\label{eq:L-1+-basis} \\
\end{align}

and the boundary conditions for the reflecting conditions:

\begin{align}
B_{1}  &\equiv \begin{pmatrix}
(1 + \underline{\xi} \Delta^{-1}_{1,-}) &0&\dots&0&0\\
0&0&\dots&0&0\\
\vdots&\vdots&\ddots&\vdots&\vdots\\
0&0&\cdots&0&0\\
0&0&\cdots&0&0
\end{pmatrix}_{M\times M} \\
B_{M}  &\equiv \begin{pmatrix}
0 &0&\dots&0&0\\
0&0&\dots&0&0\\
\vdots&\vdots&\ddots&\vdots&\vdots\\
0&0&\cdots&0&0\\
0&0&\cdots&0&(1 - \overline{\xi} \Delta^{-1}_{M,+})
\end{pmatrix}_{M\times M}
\end{align}

\subsubsection{Regular grids}
For regular grids with the uniform distance of $\Delta > 0$, \eqref{eq:L-1-regular} and \eqref{eq:L-2-regular} can be represented by

\begin{align}
L_{1-} &= \dfrac{1}{\Delta} U_1^{-} - B_1 \\
L_{1+} &= \dfrac{1}{\Delta} U_1^{+} + B_{M} \\
L_2 &= \dfrac{1}{\Delta^2} (U_1^+ - U_1^-) + B_1 + B_{M}
\end{align}

\subsubsection{Irregular grids}
For notational brevity, for vectors with the same size, $x_1, x_2$, define $x_1 x_2$ as the elementwise-multiplied vector. Then, we have
\begin{align}
L_{1-} &= \text{diag}(\Delta_{-} )^{-1} U_1^{-} - B_1 \\
L_{1+} &= \text{diag}(\Delta_{+} )^{-1} U_1^{+} + B_{M} \\
L_2 &= \text{diag} \left[ \frac{1}{2} ( \Delta_+ + \Delta_- ) \Delta_+ \right]^{-1}  U_1^{+} - 
 \text{diag} \left[ \frac{1}{2} ( \Delta_+ + \Delta_- ) \Delta_- \right]^{-1}  U_1^{-} 
+ B_1 + B_{M} 
\end{align}
We can simplify this expression further by introducing a new notation. Let $x^{-1}$ be defined as the elementwise inverse of a vector $x$ that contains no zero element. Then, $L_2$ can be represented as
\begin{align}
L_2 &=
2\left[ \text{diag} \left( ( \Delta_+ + \Delta_- )^{-1} \Delta_+^{-1} \right) U_1^{+} - 
\text{diag} \left( ( \Delta_+ + \Delta_- )^{-1} \Delta_-^{-1} \right) U_1^{-}  \right]
+ B_1 + B_{M} \\ \label{eq:L-2-by-basis}
&= 2 \text{diag} \left( ( \Delta_+ + \Delta_- )^{-1} \right) \left[ \text{diag} \left(  \Delta_+^{-1} \right) U_1^{+} - 
\text{diag} \left(  \Delta_-^{-1} \right) U_1^{-}  \right]
+ B_1 + B_{M}
\end{align}


The diagonal elements of \eqref{eq:L-2-by-basis} are also identical with the one provided in \eqref{eq:L-2} -- to see this, note that the diagonal elements of \eqref{eq:L-2-by-basis}, modulo $B_1$ and $B_{M}$, are
\begin{align}
-2 \left[ (\Delta_+ + \Delta_-)^{-1} \Delta_+^{-1} + (\Delta_+ + \Delta_-)^{-1} \Delta_-^{-1} \right] &= -2 (\Delta_+ + \Delta_-)^{-1}  ( \Delta_+^{-1} + \Delta_-^{-1} ) \\
&= -2(\Delta_+ + \Delta_-)^{-1} (\Delta_+^{-1} \Delta_-^{-1}) (\Delta_+ + \Delta_- )  \\
&= -2 (\Delta_+^{-1} \Delta_-^{-1})
\end{align}
which is identical with $\text{diag} (L_2)$ with $L_2$ from \eqref{eq:L-2} except the first row and last row that are affected by $B_1$ and $B_{M}$.

\section{Boundary conditions with operators on extended grids}
Boundary conditions can be applied manually by using operators on extended grids. This can be done by first extending $x = \{x_i\}_{i=1}^M$ to $ \overline{x} = \{x_i\}_{i=0}^{M+1}$ such that $x_1 - x_0 = \Delta_{1,+} (= \Delta_{1,-})$ and $x_{M+1} - x_M = \Delta_{M,-} (= \Delta_{M,+})$. We call $x_0$ and $x_{M+1}$, the extra nodes just before and after $x_{\min}$ and $x_{\max}$, as ghost nodes. Likewise, define $v(\overline{x})$ as $(M+2)$-vector whose $i$th element is $\overline {x}_i$. We can then define the following operators on $\overline{x}$:


\begin{equation}\label{eq:L-1-minus-extended}
\overline{L}_{1-} \equiv\begin{pmatrix}
-\Delta_{1,-}^{-1}&\Delta_{1,-}^{-1}&0&\dots&0&0&0\\
0&-\Delta_{2,-}^{-1}&\Delta_{2,-}^{-1}&\dots&0&0&0\\
\vdots&\vdots&\vdots&\ddots&\vdots&\vdots&\vdots\\
0&0&0&\dots&\Delta_{M-1,-}^{-1}&0&0\\
0&0&0&\cdots&-\Delta_{M,-}^{-1}&\Delta_{M,-}^{-1}&0
\end{pmatrix}_{M\times (M+2)}
\end{equation}

\begin{equation}\label{eq:L-1-plus-extended}
\overline{L}_{1+} \equiv \begin{pmatrix}
0&-\Delta_{1,+}^{-1}&\Delta_{1,+}^{-1}&\dots&0&0&0\\
0&0&-\Delta_{2,+}^{-1}&\dots&0&0&0\\
\vdots&\vdots&\vdots&\ddots&\vdots&\vdots&\vdots\\
0&0&0&\dots&-\Delta_{M-1,+}^{-1}&\Delta_{M-1,+}^{-1}&0\\
0&0&0&\cdots&0&-\Delta_{M,+}^{-1}&\Delta_{M,+}^{-1}&
\end{pmatrix}_{M\times (M+2)}
\end{equation}


\begin{equation}\label{eq:L-2-extended} \small
\overline{L}_2 \equiv \frac{1}{\Delta^2}\begin{psmallmatrix}
\Delta_{1,-}^{-2} &-2\Delta_{1,-}^{-1} \Delta_{1,+}^{-1}  & \Delta_{i,+}^{-2}&\dots&0&0&0\\
0&2(\Delta_{2,+}+\Delta_{2,-})^{-1} \Delta_{2,-}^{-1} &-2\Delta_{2,-}^{-1} \Delta_{2,+}^{-1} &\dots&0&0&0\\
\vdots&\vdots&\vdots&\ddots&\vdots&\vdots&\vdots\\
0&0&0&\dots&-2\Delta_{M-1,-}^{-1} \Delta_{M-1,+}^{-1}  & 2 (\Delta_{M-1,+}+\Delta_{M-1,-})^{-1} \Delta_{M-1,+}^{-1}&0\\
0&0&0&\cdots&\Delta_{M,-}^{-2} &-2\Delta_{M,-}^{-1} \Delta_{M,+}^{-1}  & \Delta_{M,+}^{-2}
\end{psmallmatrix}_{M\times (M+2)}
\end{equation}


Suppose that we want to solve a system $L v({x}) = f(x) $ for $v(x)$ where $L$ is a linear combination of discretized differential operators for some $f(x)$ that represents the values of a function $f$ on discretized $x$. To solve the system under boundary conditions on $v$, one can construct and solve the following extended system:

\begin{equation}\label{eq:extended-system}
\begin{bmatrix}
\overline{L} \\
B
\end{bmatrix} 
v(\overline{x}) = 
\begin{bmatrix}
f(x) \\
b
\end{bmatrix} 
\end{equation}

with $M_E$ by $(M+2)$ matrix $B$ and $M_E$-length vector $b$ that represent the current boundary conditions, where $M_E$ is the number of boundary conditions to be applied. 

\subsection{Reflecting barrier conditions}

To apply reflecting barrier conditions $v'(x_{\min}) = v'(x_{\max}) = 0$, one can use

\begin{equation}\label{eq:reflecting-barrier-matrix}
B = \begin{bmatrix}
1 & -1 & 0 & \dots & 0 & 0 & 0 \\
0 & 0 & 0 & \dots & 0 & -1 & 1\\
\end{bmatrix}_{2 \times (M+2)} \quad 
b = \begin{bmatrix}
0 \\
0
\end{bmatrix}
\end{equation}

\section{Applications}

\subsection{Hamilton–Jacobi–Bellman equations (HJBE)}
Consider solving for $v$ from the following optimal control problem
\begin{align}
v(x_0) = \max_{ {\{\alpha(t) \} }_{t \geq 0} } \int_{0}^\infty e^{-\rho t} r( x(t), \alpha(t )) dt
\end{align}

with the law of motion for the state 
\begin{align}
dx = \mu dt + \sigma dW 
\end{align}


for some constant $\mu \geq 0$ and $\sigma \geq 0$ with $x(0) = x_0$.

Let $\alpha^*(t)$ be the optimal solution. Suppose that $r$ under $\alpha^*(t)$ can be expressed in terms of state variables, $r^* (x)$. Then, the HJBE yields

\begin{equation}\label{eq:hamilton-jacobi-bellman}
\rho v(x) = r^*(x) +  \mu  \partial_{x} v(x) + \dfrac{\sigma^2}{2} \partial_{xx} v(x)
\end{equation}

In terms of differential operators, one can rewrite the equation as
\begin{equation}\label{eq:hjbe-system-function}
\tilde{L}  v(x) = r^*(x)
\end{equation}

with $\tilde{L} = \rho -  \tilde{L}_x$ where $\tilde{L}_x$ is defined as

\begin{equation}\label{eq:L-defn}
\tilde{L}_x = \mu \partial_{x} + (\sigma^2/2) \partial_{xx}
\end{equation}


By descretizing the space of $x$, one can solve the corresponding system by using discretized operators for $\partial_{x}$ ($L_{1+}$), $\partial_{xx}$ ($L_2$) on some grids of length $M$, $\{x_i\}_{i=1}^M$:
\begin{align}\label{eq:L-descritized-defn}
L_x = \mu L_{1+} + \dfrac{\sigma^2}{2} L_{2}
\end{align}
so that $v$ from \eqref{eq:hjbe-system-function} can be computed by solving the following discretized system of equations:
\begin{align}
L v &= r^*
\end{align}
where $v$ and $r^*$ are $M$-vectors whose $i$th elements are $v(x_i)$ and $r^*(x_i)$, respectively, and $L$ is defined as $L = \rho I - L_x$.



\subsection{Kolmogorov forward equations (KFE) under diffusion process}
Let $g(x, t)$ be the distribution of $x$ at time $t$ from the example above. By the Kolmogorov forward equation, the following PDE holds:

\begin{equation}\label{eq:kfe}
\partial_{t} g(x, t) = - \mu \partial_{x}  g(x,t) + \dfrac{\sigma^2}{2} \partial_{xx} g(x,t)
\end{equation}

\subsubsection{Stationary distributions}
The stationary distribution $g^*(x)$ satisfies

\begin{equation}\label{eq:kfe-stationary}
0 = - \mu \partial_{x} g^*(x) + \dfrac{\sigma^2}{2} \partial_{xx} g^*(x)
\end{equation}

which can be rewritten as 

\begin{equation}
\tilde{L}^*_x g(x) = 0
\end{equation}

where 

\begin{equation}\label{eq:L_star-defn}
\tilde{L}^*_x =  - \mu \partial_{x} + (\sigma^2/2) \partial_{xx}
\end{equation}

By descretizing the space of $x$, one can solve the corresponding system by using discretized operators for $\tilde{L}^*_x$. Note that the operator for the KFE in \eqref{eq:L_star-defn} is the adjoint operator of the operator for the HJBE in \eqref{eq:L-defn}, and the correct discretization scheme for \eqref{eq:L_star-defn} can be, analogously, done by taking the transpose of the discretized operator for HJBE in \eqref{eq:L-descritized-defn} -- see \cite{gabaix16} and \cite{achdou17}. Hence, one can find the stationary distribution by solving the following discretized system of equations:

\begin{equation}
L^T_x g = 0 
\end{equation}
where $L^T_x$ is the transpose of $L_x$ from \eqref{eq:L-descritized-defn} and $g$ is an $M$-vector whose element is $g(x_i)$ such that $\sum_{i=1}^M g(x_i) = 1$.

\subsubsection{Full dynamics of distributions}
One can also solve the full PDE in \eqref{eq:kfe}, given an initial distribution $g(x, 0)$. After discretization, note that \eqref{eq:kfe} can be rewritten as
\begin{equation}\label{eq:kfe-discretized}
\dot{g}(t) = L^T_x g(t)
\end{equation}
where $\dot{g}(t)$ is an $M$-vector whose $i$th element is $\partial_{t} g(x_i, t)$, which can be efficently solved by a number of differential equation solvers available in public, including \cite{rackauckas17}.

\newpage
\bibliography{hact}
% The notebook is based on my previous version that can be found at https://github.com/ubcecon/computing_and_datascience/blob/a33d14191afba2d13598b331b3623a9512dccc90/continuous_time_methods/notes/discretized-differential-operator-derivation.tex by @chiyahn
\end{document}
